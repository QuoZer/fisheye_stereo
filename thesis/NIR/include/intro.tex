За последние годы был совершён существенный прогресс в доступности и точности сенсоров, позволяющих мобильным роботам 
осуществлять оценку окружающего пространства. Такие информационно-измерительные устройства как лидары, сонары и стереокамеры
 стали основной опорой для алгоритмов автономной навигации и локализации. Тем не менее в роботах по-прежнему 
присутствуют телевизионные системы, так как они дают наиболее легко воспринимаемую информацию для оператора в случаях, когда 
его вмешательство необходимо. 
Число и расположение видеокамер может быть различным. Бортовые телевизионные системы классифицируются по различным 
критериям\cite{varlashin}:
\begin{itemize} 
    \item по количеству камер (монокулярные или многокамерные).
    \item по функциональному назначению:
    \begin{enumerate}[leftmargin=12mm]  %\setlength{\leftmargin}{10mm}
        \item одиночные камеры;
        \item PTZ-камеры (англ. Pan-Tilt-Zoom);
        \item стереокамеры;
        \item системы кругового обзора;
        \item комбинированные системы.
    \end{enumerate}
\end{itemize}

Для наиболее эффективного покрытия телевизионной системой максимального объёма окружающего пространства в ней могут использоваться 
камеры "рыбий глаз" (англ. fisheye-camera), позволяющие одним кадром покрыть угловое поле свыше $180^\circ$.  
Набор таких камер может составлять систему кругового обзора, позволяющую оператору видеть не только в любом направлении,                
но даже с видом от третьего лица \cite{birdeye}. 
    
В случае автономных мобильных роботов телевизионный круговой обзор нужен лишь по необходимости, но при этом может быть весьма 
дорогостоящими и занимать место в корпусе. При этом в случае наличия пересечений полей зрения камер, входящих в систему, возможно 
применение принципа стереозрения для построения карты окружающего пространства. Однако значительные радиальные искажения изображения, вызванные особенностями используемых объективов,
 не позволяют использовать известные алгоритмы стереозрения без обработки изображений.                                   
 
  Реализация системы стереозрения на основе ортогонально ориентированных камер, способной эффективно задействовать особенности 
  сверхширокоугольных объективов, позволит дать существующим роботам новый способ получать информацию об окружении.

Целью работы является разработка и исследование точности системы стереозрения, основанной на ортогонально расположенных сверхширокоугольных камерах.

В ходе работы решаются следующие задачи:
\begin{itemize}     % TODO: уточнить
    \item обзор современных систем стереозрения, использующих изображения с широкоугольных камер,
    \item обзор моделей, применяющихся в моделировании камер с объективами "рыбий глаз",
    \item обоснование выбора программного обеспечения, используемого для разработки и тестирования алгоритма, 
    \item разработка алгоритма устранения искажений fisheye-изображения в  области кадра, % FIXME: этот неймдропинг надо где-то раньше сделать
    \item моделирование системы  стереозрения,                                      
    \item определение точности оценки глубины системой.  
\end{itemize}


\clearpage

За последние годы был совершён существенный прогресс в доступности и точности сенсоров, позволяющих 
мобильным роботам осуществлять оценку окружающего пространства. Такие информационно-измерительные устройства
как лидары, сонары (и что-нибудь ещё) стали основной опорой алгоритмов для алгоритмов автономной навигации и локализации.
Тем не менее в роботах по-прежнему присутствуют оптические системы, так как они дают наиболее читаемую информацию для 
оператора в случаях, когда его вмешательство необходимо.   Существенная часть современных мобильных роботов имеют у себя на 
борту камеры с широким () или сверхшироким углом обзора ().  \cite{Hartley2004}
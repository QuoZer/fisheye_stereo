За последние годы был совершён существенный прогресс в доступности и точности сенсоров, позволяющих мобильным роботам 
осуществлять оценку окружающего пространства. Такие информационно-измерительные устройства как лидары, сонары и стереокамеры
 стали основной опорой алгоритмов для алгоритмов автономной навигации и локализации. Тем не менее в роботах по-прежнему 
присутствуют оптические системы, так как они дают наиболее читаемую информацию для оператора в случаях, когда его вмешательство 
необходимо. Многие современные мобильные роботы имеют у себя на борту камеры с сверхшироким ($ > 90^\circ $)                      
углом зрения, так как они, хоть и вносят искажения в воспринимаемую картину, позволяют охватить больше окружающего пространства.               %FIXME:  Формулировки. 
Набор таких камер может составлять систему кругового обзора, позволяющую оператору видеть не только в любом направлении,                %TODO: цитировать систему кругового обзора
но даже с видом от третьего лица \cite{birdeye}. 
    
В случае автономных мобильных роботов подобные системы включаются лишь по необходимости, но при этом могут быть весьма 
дорогостоящими и занимать место в корпусе. Широкоугольные камеры в системах кругового обзора роботов часто имеют области                     % FIXME: 'часто' - оценочный термин?
пересечения полей зрения, что позволяет проводить оценку глубины. Реализация системы стереозрения, способной работать в таких
конфигурациях, позволит дать существующим роботам новый способ получать информацию об окружении и проектировать будущих роботов 
с учётом этой возможности. 

Однако значительные радиальные искажения изображения, вызванные особенностями используемых объективов, вместе с тем фактом, что 
области пересечения обычно расположены ближе к краям изображения, не позволяют использовать известные алгоритмы стереозрения.                                       % FIXME:не совсем алгоритмы стереозрения, скорее калибровки. Дополнить абзац

Целью работы является разработка и изучение точности системы стереозрения, основанной на ортогонально расположенных сверхширокоугольных камерах.

В ходе работы решаются следующие задачи:
\begin{itemize}     % TODO: уточнить
    \item Обзор современных систем стереозрения, использующих изображения с широкоугольных камер.
    \item Обзор моделей, применяющихся в моделировании камер с объективами "рыбий глаз".
    \item Обоснование выбора программного обеспечения, используемого для разработки и тестирования алгоритма. 
    \item Разработка алгоритма устранения искажений fisheye-изображения в любой области кадра. % FIXME: этот неймдропинг надо где-то раньше сделать
    \item Моделирование системы  стереозрения в виртуальной среде.                                    
    \item Определение точности оценки глубины стереосистемой.  
\end{itemize}


\clearpage
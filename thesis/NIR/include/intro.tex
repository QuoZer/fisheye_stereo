
Получение трёхмерной структуры пространства по стереоснимкам - это задача, первые решения которой
были получены десятилетия назад. Ранние работы фокусировались в основном на способах поиска соответствий
и геометрических основах, лежащих в основе процесса. Существенный объём научной работы продолжает
 производиться в области стереозрения и по сей день. Был достигнут заметный прогресс в повышении точности 
 результатов и понижении вычислительных мощностей, требуемых для их достижения, однако эти области остаются 
 фокусом исследований. 

Улучшение точности и производительности алгоритмов является нетривиальной задачей. На точность 
полученных результатов оказывает влияние нехватка информации, вызванная заслонением объектов, наличием наклонных
плоскостей и другими факторами, влияющими на сложность восстановления трёхмерных объектов. Разрешение
сенсоров также растёт с каждым годом, увеличивая вычислительную сложность поиска соответствий на кадрах с 
каждой камеры. Таким образом, исследователи в области стереозрения пытаются найти компромисс между этими
 двумя характеристиками. Однако для каждого конкретного алгоритма этот компромисс может быть смещён в 
 ту или иную сторону. \cite{Hartley2004}
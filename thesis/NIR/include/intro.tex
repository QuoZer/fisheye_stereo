За последние годы был совершён существенный прогресс в доступности и точности сенсоров, позволяющих мобильным роботам 
осуществлять оценку окружающего пространства. Такие информационно-измерительные устройства как лидары, сонары (и что-нибудь                     %TODO: что-нибудь ещё
ещё) стали основной опорой алгоритмов для алгоритмов автономной навигации и локализации. Тем не менее в роботах по-прежнему 
присутствуют оптические системы, так как они дают наиболее читаемую информацию для оператора в случаях, когда его вмешательство 
необходимо. Существенная часть современных мобильных роботов имеют у себя на борту камеры с широким () или сверхшироким ()                      %TODO: найти значения углов (есть в лекциях)
углами обзора, так как они, хоть и вносят искажения в воспринимаемую картину, позволяют охватить больше окружающего пространства.               %FIXME:  Формулировки. 
Набор таких камер может составлять систему кругового обзора \cite{}, позволяющую оператору видеть не только в любом направлении,                %TODO: цитировать систему кругового обзора
но даже с видом от третьего лица \cite{}. Схема подобной системы приведена на рисунке \ref{pic:examples}. 

\begin{figure}[H]
    \begin{center}
        \includegraphics[scale=0.5]{pics/sample.png}                                                                                            %TODO: найти/нарисовать схему с областями зрения камер
        \caption{Системы кругового обзора}
        \label{pic:examples}
    \end{center}
\end{figure}
    
В случае автономных мобильных роботов подобные системы включаются лишь по необходимости, но при этом могут быть весьма 
дорогостоящими и занимать место в корпусе.                                                                                                      % TODO: уточнить про стоимость. Перефразировать предложение. ценное место в корпусе? 
Согласно схемам на рисунке \ref{pic:examples} широкоугольные камеры в системах кругового обзора роботов часто имеют области                     % FIXME: 'часто' - оценочный термин?
пересечения их полей зрения, что позволяет проводить оценку глубины / использовать алгоритмы стереозрения / использовать камеры как стереопару. % FIXME: выбрать хорошую формулировку
Это даёт роботу вспомогательный (или единственный) источник трёхмерной информации об окружении без дополнительных расходов.                     % FIXME: разве что на вычислитель, вероятно...

Однако значительные радиальные искажения изображения, вызванные особенностями используемых объективов, вместе с тем фактом, что области пересечения 
обычно расположены ближе к краям изображения, не позволяют использовать известные алгоритмы стереозрения.                                       % FIXME:не совсем алгоритмы стереозрения, скорее калибровки. Дополнить абзац

Целью работы является разработка и изучение точности системы стереозрения, основанной на ортогонально расположенных сверхширокоугольных камерах.

В ходе работы решаются следующие задачи:
\begin{itemize}
    \item Обзор современных алгоритмов стереозрения / алгоритмов калибровки изображений широкоугольных камеры.
    \item Обоснование выбора программного обеспечения, используемого для разработки и тестирования алгоритма. 
    \item Разработка алгоритма устранения искажений fisheye-объектива и системы стереозрения на его основе.                                     % FIXME: этот неймдропинг надо где-то раньше сделать
    \item Оценка точности оценки глубины в виртуальной среде.  
\end{itemize}

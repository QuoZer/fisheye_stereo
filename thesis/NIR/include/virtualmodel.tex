\subsection{Обоснование выбора ПО}

Разработку и первоначальные испытания алгоритма стереозрения целесообразно проводить в виртуальной среде. Это позволяет значительно упростить разработку, 
так как уменьшает время на проверку гипотез и расходы на реальное оборудование, особенно в случае неудачных испытаний. Из-за этих факторов виртуальное моделирование 
в робототехнике приобрело широкое распространение и активно применяется, например, для разработки систем локализации и навигации беспилотного транспорта \cite{}. 
Возросшее качество компьютерной графики к тому же позволило моделировать реалистичное окружение, что особенно важно при работе с системами технического зрения. 

Для разработки алгоритма, описанного в этой работе, нужна виртуальная среда, в которой можно симулировать несколько широкоугольных камер и настраивать их параметры, 
легко интегрировать алгоритмы технического зрения и создать окружение, приближенное к тому, в котором будет работать алгоритм. На данный момент исследователю доступен 
широкий выбор программного обеспечения, подходящего для этой задачи. В таблице \ref{tab:sims} представлено сравнение имеющихся предложений по основным изложенным выше 
требованиям.

\begin{table}[]             % TODO: Сыровато. Надо дополнить. Проверить ГОСТовость подписи  FIXME: Таблица не умещается, но вращать не хочется. 
    \caption{Сравнение ПО для симуляции }
    \label{tab:sims}
    \begin{tabular}{|l|l|l|l|l|l|}
    \hline
    \textbf{Название} & \textbf{\begin{tabular}[c]{@{}l@{}}Симуляция \\ fisheye-камер\end{tabular}} & \textbf{\begin{tabular}[c]{@{}l@{}}Реалистичное \\ моделирование\end{tabular}} & \textbf{Интеграция кода}                                            & \textbf{Доступность} & \textbf{Примечания} \\ \hline
    Gazebo            & Возможна                                                                    & Затруднено                                                                     & \begin{tabular}[c]{@{}l@{}}Возможна \\ посредством ROS\end{tabular} & Бесплатно            &                     \\ \hline
    RoboDK            & Нет                                                                         & Затруднено                                                                     & Нет                                                                 & От 145€              &                     \\ \hline
    Webots            & Затруднена                                                                  & Возможно                                                                       & Возможна                                                            & Бесплатно            &                     \\ \hline
    CoppeliaSim       & Затруднена                                                                  & Затруднено                                                                     & Возможна                                                            & Бесплатно            &                     \\ \hline
    NVIDIA Isaac Sim  & Возможна                                                                    & Возможно                                                                       & Возможна                                                            & Бесплатно            &                     \\ \hline
    CARLA             & Затруднена                                                                  & Возможно                                                                       & Возможна                                                            & Бесплатно            &                     \\ \hline
    Unity             & Возможна                                                                    & Возможно                                                                       & Возможна                                                            & Бесплатно            &                     \\ \hline
                      &                                                                             &                                                                                &                                                                     &                      &                     \\ \hline
    \end{tabular}
\end{table}

По результатам оценки собранные сведений было принято решение проводить разработку в симуляторе Unity. Он позволяет подробно настраивать камеру и эмулировать fisheye-объектив, 
строить реалистичные сцены благодаря свободному импорту моделей, а при программировании в симуляторе можно использовать сторонние программы в виде динамически подключаемых библиотек. 
По функционалу так же подходит NVIDIA Isaac Sim, но от него пришлось отказаться из-за высоких системных требований и новизны продукта.     % TODO: звучит тупо

Разрабатываемое решение должно иметь возможность внедрения в ПО робота, % TODO: не совсем ПО всё-таки
поэтому должно реализовываться на одном из популярных и быстродейственных языков программирования. Учитывая необходимость интеграции с Unity и потребность использовать популярные библиотеки
, был выбран язык C++.  Другим важным фактором является основная библиотека обработки изображений. В качестве основы для программной части была выбрана библиотека OpenCV, являющаяся стандартом 
при разработке систем технического зрения. Она доступна к использованию со множеством языков программирования, но наилучшую производительность показывает именно с C++ \cite{}.                              % мб есть сравнение
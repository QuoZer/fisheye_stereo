\subsection{Обоснование выбора ПО}

Разработку и первоначальные испытания алгоритма стереозрения целесообразно проводить в виртуальной среде. Это позволяет значительно упростить разработку, 
так как уменьшает время на проверку гипотез и расходы на реальное оборудование, особенно в случае неудачных испытаний. Из-за этих факторов виртуальное моделирование 
в робототехнике приобрело широкое распространение и активно применяется, например, для разработки систем локализации и навигации беспилотного транспорта \cite{}. 
Возросшее качество компьютерной графики к тому же позволило моделировать реалистичное окружение, что особенно важно при работе с системами технического зрения. 

Для разработки алгоритма, описанного в этой работе, нужна виртуальная среда, в которой можно симулировать несколько широкоугольных камер и настраивать их параметры, 
легко интегрировать алгоритмы технического зрения и создать окружение, приближенное к тому, в котором будет работать алгоритм. На данный момент исследователю доступен 
широкий выбор программного обеспечения, подходящего для этой задачи. В таблице \ref{table:sims} представлено сравнение имеющихся предложений по основным изложенным выше 
требованиям.
\subsection{Описание системы стереозрения}

Предлагаемая система стереозрения состоит из двух или более камер с объективами типа "рыбий глаз" $\geqslant 180^\circ$,
расположенных ортогонально. В таком случае поля зрения соседних камер могут пересекаться, 
образуя  области, точки пространства в которых видны на обеих камерах. Это позволяет получить информацию о 
глубине в указанных областях. % FIXME: Всё описание просто бггг

Пример такой системы для робота "Капитан" представлен на рисунке \ref{pic:4cam_system}. Здесь 4 камеры, изображённых
 полукругами, с центрами в точках $C_{0-3}$ размещены спереди, сзади и по бортам корпуса робота. Пересечения их полей зрения 
 образуют 4 области объёмного зрения, закрашенных голубым.    % TODO: дополнить описание картинки (и нарисовать саму картинку) 
 
\addimghere{sample_4cam}{0.7}{Система стереозрения из четырёх камер на примере робота "Капитан"}{pic:4cam_system} 

Применение традиционных алгоритмов стереозрения предполагает наличие стереопары - двух камер с известным взаимным положением,
наблюдающих  одну область пространства с разных ракурсов. Описанный в предыдущей секции метод  устранения искажений 
сверхширокоугольной камеры позволяет создавать виртуальные камеры-обскуры и регулировать их направление обзора. Таким образом,
стереопару можно сформировать из двух таких виртуальных камер.  %параллельно биссектрисе угла, образованного пересечением полей зрения fisheye-камер. 
Далее для упрощения рассмотрения системы будет считаться, что оптические оси всех камер находятся в одной плоскости, 
а на ориентацию виртуальных камер влияет только один угол. % FIXME: Какой? ...только угол в этой плоскости? ...угол рыскания? 

На рисунке \ref{pic:2cam_scheme} изображён вариант системы с двумя камерами под углом $90^\circ$, соответствующий фрагменту схемы,     
  представленной на рисунке \ref{pic:4cam_system}. 

\addimghere{sample_simple2cam}{0.7}{Геометрическая модель бинокулярной системы стереозрения}{pic:2cam_scheme} %TODO: перерисовать схему?

  Здесь у полей зрения камер $C_0$ и $C_1$ есть область пересечения, обозначенная красным. 
Эта область эквивалентна области пересечения полей зрения двух камер с полями зрения $90^\circ$ (обозначены
оранжевым), повёрнутых на $45^\circ$ в сторону области интереса. Примеры изображений, полученных в такой конфигурации,
изображены на рисунке \ref{pic:dewarped_exmples}. 

\addimghere{sample}{0.7}{Пример исходных изображений и снимков виртуальной стереопары}{pic:dewarped_exmples}


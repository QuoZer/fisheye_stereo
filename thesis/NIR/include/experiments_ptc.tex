
\subsection{Исследование точности оценки глубины}  % понятие облака точек не введено, поверхность какая? 

В качестве целевой поверхности выбрана виртуальная плоскость с нанесённой на неё нерегулярной текстурой высоко разрешения. Размеры и
положение плоскости относительно камер известно с высокой точностью, что позволяет сравнить результаты стереореконструкции
с реальным положением целевого объекта и оценить ошибку. В качестве метрики оценки выбрано среднее квадратичное отклонение
положения точек от модели плоскости. 

Исследования проведены в MATLAB, снимки получены с помощью виртуальной модели при  одинаковых условиях освещённости и 
постоянных  настройках всех  компонентов системы.  Ширина базы $B$ выбрана равной 1  метру.  Сначала для обеих стереопар 
проведена калибровка по снимкам узора шахматной доски с помощью методики \cite{stereo_calib}, что позволило получить 
внутренние и внешние параметры камер стереопары. % FIXME: ну вообще они у нас как бы есть. Мб сравнить ещё оценки с реальными 
Далее происходит построение карты расхождений методом полу-глобального  сопоставления \cite{SGBM} и 3D-реконструкция сцены. 
Результата реконструкции представлен  в виде облака точек,  изображённого на рисунке \ref{pic:raw_pointcloud}.

\addimghere{pointcloud}{0.7}{Неочищенное облако точек}{pic:raw_pointcloud}
\pdfmargincomment{Три вида с одного раркурса}
% TODO: сюда можно и картинку после обработки поинтклауда \addtwoimghere{}{}{}{}{}
По заданным параметрам строится модель целевой плоскости, а точки за пределами её окрестности отбрасываются. % FIXME: размеры окрестности
Затем скрипт перебирает все оставшиеся точки и рассчитывает среднее квадратичное отклонение по длине нормали от 
точки к плоскости. 
Описанный алгоритм применён к снимкам целевой плоскости на разных расстояниях с эталонной и исследуемой стереопар.
Результаты приведены на графике \ref{plot:MSE_compar}. На графике приведены значения среднеквадратичной ошибки и
дисперсия для исследуемой и эталонной стереопар при разном удалении целевого объекта.

\addimghere{alt1mErrorGraph}{1.0}{График среднеквадратичной ошибки и  дисперсии в зависимости от расстояния для эталонной и исследуемой стереопар}{plot:MSE_compar}

Как видно из аппроксимирующих прямых на графике, среднеквадратичная ошибка оценки поверхности в исследуемой системе
 в 3 раза больше этого показателя для традиционной стереопары. Меньшая точность исследуемой системы связана с влиянием 
на алгоритм стереосопоставления дефектов изображения, описанных в предыдущей секции. Кроме того, с увеличением расстояния
уменьшается разрешение, приходящееся на поверхность, и, соответственно, количество точек с подсчитанной глубиной. 
Тем не менее она остаётся в пределах $3\%$ от реального расстояния.  % FIXME: хммммм, как-то уж больно хорошо. Или  плохо... 10 см на 5 метрах


%
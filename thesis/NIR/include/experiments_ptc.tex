
\subsection{Оценка отклонения облака точек от поверхности}  % понятие облака точек не введено, поверхность какая? 

В качестве целевой поверхности выбрана виртуальная плоскость с нанесённой на неё текстурой высоко разрешения. Размеры и
положение плоскости относительно камер известно с высокой точностью, что позволяет сравнить результаты стереореконструкции
с реальным положением целевого объекта и оценить ошибку. В качестве метрики оценки выбрано среднее квадратичное отклонение
положения точек от модели плоскости. 

Исследования проведены в MATLAB, снимки получены с помощью виртуальной модели. Сначала для обеих стереопар проводится
калибровка по снимкам узора шахматной доски с помощью методики \cite{stereo_calib}, что позволяет получить 
внутренние и внешние параметры камер. % FIXME: ну вообще они у нас как бы есть. Мб сравнить ещё оценки с реальными 
Далее происходит построение карты расхождений \cite{SGBM} и 3D-реконструкция сцены. Результирующее облако точек 
представлено на рисунке
\ref{pic:raw_pointcloud}.

\addimghere{pointcloud}{0.7}{Неочищенное облако точек}{pic:raw_pointcloud}
% TODO: сюда можно и картинку после обработки поинтклауда \addtwoimghere{}{}{}{}{}
По заданным параметрам строится модель целевой плоскости, а точки за пределами её окрестности отбрасываются. % FIXME: размеры окрестности
Затем скрипт перебирает все оставшиеся точки и рассчитывает среднее квадратичное отклонение по длине нормали от 
точки к плоскости. Код скрипта приведён в приложении \hyperlink{app:matlab}{Д}.

Описанный алгоритм применён к снимкам целевой плоскости на разных расстояниях с эталонной и исследуемой стереопар.
Результаты приведены на графике \ref{plot:MSE_compar}.

\addimghere{error_plot}{1.0}{График среднеквадратичной ошибки в зависимости от расстояния для эталонной и исследуемой стереопар}{plot:MSE_compar}

Как видно из графика, ошибка оценки поверхности в исследуемой системе в среднем в два раза больше этого показателя для
традиционной стереопары. Меньшая точность исследуемой системы связана с влиянием факторов, описанных в предыдущей секци,
на алгоритм стереосопоставления.  Тем не менее она остаётся в пределах $3\%$ от реального расстояния.  % FIXME: хммммм, как-то уж больно хорошо. Или  плохо... 10 см на 5 метрах


%
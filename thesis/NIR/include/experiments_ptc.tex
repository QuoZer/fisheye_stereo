\subsection{С чем сравниваем}  % понятие облака точек не введено, поверхность какая? 

Результатом работы алгоритма стереозрения является карта глубины воспринимаемого пространства. Подобные карты могут
использоваться, например, алгоритмами навигации и локализации для построения карты окружения робота. Точность работы 
этих алгоритмов зависит от того, насколько точно карта глубины передаёт реальную информацию о форме объектов. Оценить 
эту характеристику для отдельной системы стереозрения проблематично, так как она зависит от множества факторов...
Поэтому оценка пригодности для применения системы в реальности проведена в сравнении с виртуальной моделью традиционной
стереопары.

Виртуальная модель позволяет разместить сразу несколько камер в одной точке пространства, таким образом возможно
в существующую модель в Unity добавить ещё 2 камеры, совпадающие по параметрам и положению с виртуальными камерами на 
рисунке \ref{pic:2cam_scheme}. Разрешение этих камер выбрано исходя из размеров проекции $\nu_i$ области интереса на широкоугольном
снимке. Для камер "рыбий глаз" с разрешением $1080*1080$ пикселей она составляет 287482 пикселей, что аналогично камере 
с разрешением $540*540$.  В результате получена эталонная стереопара для сравнения. % TODO: чувствуется какая-то недосказанность



Обе стереопары были откалиброваны по снимкам узора шахматной доски с помощью методики \cite{zhang}, реализованной в 
Matlab. 


\subsection{Оценка отклонения облакка точек от поверхности}  % понятие облака точек не введено, поверхность какая? 



%MSE			
%4m	5m	7.5m	10m	12.5m	15m
%Ref	0,4	0,500	0,750	1,000	1,250	1,500
%FY	0,0068	0,01010	0,0154	0,02360	0,0237	0,02960
%REG	0,0043	0,00520	0,00760	0,01020	0,0081	0,01510
%1,70%	2,02%	2,05%	2,36%	1,90%	1,97%
%1,08%	1,04%	1,01%	1,02%	0,65%	1,01%


\subsection{Оценка качества устранения искажений}

Добавление в виртуальную модель эталонной камеры позволяет оценить качество устранения искажений путём сравнения
её снимков со снимками виртуальной камеры-обскуры модуля устранения искажений. % FIXME: никакой "модуль..." ранее не вводился, нужно переписать

Параметры полинома (\ref{eqn:scaramuzza}) для устранения искажений получены с помощью MATLAB Camera Calibrator и занесены 
в код библиотеки обработки изображений. % FIXME: опять-таки будто про OpenCV речь, надо как-то ссылаться на свой алгоритм\модуль
Снимки после устранения искажений и с эталонной камеры показаны на рисунке \ref{pic:2images_compar}.

\addtwoimghere{pic_fy}{0.4}{pic_reg}{0.4}{Слева - снимок после устранения искажений; справа - эталонный снимок.}{pic:2images_compar}

Визуально снимки очень похожи, однако при более детальном рассмотрении на левом изображении заметна большая резкость, вызванная, 
вероятно, округлениями чисел в процессе проекции. Также присутствуют малозаметные искажения геометрии. Более 
явно эти дефекты можно увидеть на разностном изображении, представленном на рисунке \ref{pic:difference}. 

\addimghere{difference}{0.5}{Разностное изображение (резкость увеличена). Чем светлее участок, тем сильнее различия}{pic:difference}

Анализ этого изображения подтверждает различия в резкости обеих картинок и наличие искажений преимущественно в левой части снимка. Тем не менее, искажения
достаточно несущественны и проявляются лишь близь краёв исходного изображения, что позволяет считать подобные снимки пригодными 
к применению в системе стереозрения.  
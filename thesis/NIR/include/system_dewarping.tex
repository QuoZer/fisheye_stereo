\subsection{Алгоритм устранения искажений}

Как уже упоминалось в секции \ref{camera_model}, существующие способы устранения радиальных искажений не
позволяют работать близко к краям изображений, поэтому для реализации предлагаемой системы стереозрения     % FIXME: "работать"
был разработан алгоритм устранения искажений на основе модели \cite{scaramuzza}. Схема геометрического
принципа, лежащего в основе этого алгоритма, представлена на рисунке \ref{pic:sweeping}.

\addimghere{projection_sweeping}{0.7}{Схема принципа устранения искажений}{pic:sweeping} 

Цель алгоритма - найти, куда на выходном изображении проектируются все пиксели из выбранного участка входного 
сверхширокоугольного изображения. Сделать это можно, выполнив обратное преобразование \ref{eqn:scaramuzza} для каждого пикселя 
fisheye-снимка и затем прямое преобразование модели камеры-обскуры для получения результирующей проекции. Однако такой подход 
приведёт к возникновению дефектов из-за несовпадения частот дискретизации двух изображений.                 % TODO:  скорее всего наврал. Надо проконсультироваться 
Поэтому вместо этого алгоритм выполняет обратное преобразование  для каждого пикселя итогового изображения $\nu$, 
находя таким образом соответствующую ему точку в системе координат камеры $({X_c, Y_c, Z_c})$ 
\begin{equation}
    \label{eq:uv_to_xyz}
    \left[\begin{matrix}x_c\\y_c\\z_c\\\end{matrix}\right] = \left[\begin{matrix} {(u+c_x)*z_c}/f \\  {(v+c_y)*z_c}/f \\ z_c \\\end{matrix}\right],
\end{equation} 
где $c_x, c_y$ - координаты центра изображения; $f$ - фокусное расстояние. 

Набор таких точек формирует прямоугольную область $\nu_p$ с центром в точке $O_p$ и является виртуальной камерой-обскурой        % FIXME: прямоугольник?
с оптической осью $Z_p$. Поворот точек, входящих в $\nu_p$, с помощью матрицы вращения $\bm{R}$ образует $\nu'_p$ и
 позволяет таким образом задать направление обзора и ориентацию виртуальной камеры. 
\begin{equation}
    \label{eq:sweeped}
    \left[\begin{matrix}x'_p\\y'_p\\z'_p\\\end{matrix}\right] = \left[\begin{matrix}x_p\\y_p\\z_p\\\end{matrix}\right] \bm{R}.
\end{equation}  
\begin{equation}
    \label{eq:R}
    \bm{R} = \left[\begin{matrix}\cos{\alpha}&-\sin{\alpha}&0\\\sin{\alpha}&\cos{\alpha}&0\\0&0&1\\\end{matrix}\right]\left[\begin{matrix}1&0&0\\0&\cos{\beta}&\sin{\beta}\\0&-\sin{\beta}&\cos{\beta}\\\end{matrix}\right]\left[\begin{matrix}\cos{\gamma}&0&-\sin{\gamma}\\0&1&0\\\sin{\gamma}&0&\cos{\gamma}\\\end{matrix}\right],
\end{equation} 
где $\alpha, \beta, \gamma$ - углы Эйлера. % TODO: точно ли Эйлера? 

Тогда обратная fisheye-проекция точек из $\nu'_p$ позволяют получить область $\nu_i$ исходного изображения с искомыми пикселями. 
Таким образом, зная как геометрически проектируется каждая точка из $\nu$ в $\nu_i$, можно перенести информацию о цвете и получить 
изображение с устранёнными радиальными искажениями в любой части поля зрения камеры. 

Так как в используемой модели искажения в точке считаются центрально симметричными и зависят только от её удаления от центра изображения, 
рассмотрим ход падающего луча в координатах $(Z_c, \rho)$, изображённый на рисунке \ref{pic:scara_graph}.

\addimghere{scara_graph}{0.5}{}{pic:scara_graph} 

Для модели (\ref{eqn:scaramuzza}) обратная проекция записывается как 
\begin{equation}
    \label{eq:back_scara}
    \left[\begin{matrix}u_i\\v_i\\\end{matrix}\right] = \left[\begin{matrix} \frac{x_c}{\lambda}  \\  \frac{y_c}{\lambda} \\\end{matrix}\right],
\end{equation}  
где $\lambda = \rho_c / \rho_i$ - масштабный коэффициент. 

Для нахождения $\rho_i$ был применён метод последовательных приближений. Блок-схема алгоритма приведена на рисунке \ref{pic:newton_scheme}.

\addimghere{sample}{0.5}{Блок-схема алгоритма поиска $\rho$}{pic:newton_scheme} 

Очевидно, весь процесс преобразования $\nu \rightarrow \nu_i$ требует выполнения существенного количества математических операций, что 
негативно сказывается на скорости, с которой алгоритм может обрабатывать изображения в реальном времени. Однако при неизменных параметрах 
модели алгоритм достаточно выполнить лишь один раз, записав результат в таблицу поиска - структуру данных, которая позволяет дальнейшие 
преобразования проводить по уже известным соотношениям между пикселями. Время построения таблицы поиска для выходного изображения 
разрешением 540*540 пикселей и входного изображения 1080*1080 пикселей составляет $\sim 2.2c$. % TODO: нормально замерить. Возможно посчитать O(.)  O(n), n - кол-во пикселей? 
Это время можно уменьшить, применив различные оптимизации и параллельные вычисления (проекция каждого пикселя может рассчитывать независимо). % FIXME: оптимизации это про симметричность преобразований например. Надо нормально написать
Время отображения с использованием таблиц поиска составляет для того же изображения $ ... $, что позволяет применять алгоритм для устранения 
искажений в реальном времени.
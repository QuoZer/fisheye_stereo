\subsection{Обоснование выбора ПО}
\label{sec:software}
Разработку и первоначальные испытания алгоритма стереозрения целесообразно проводить в виртуальной среде. Это позволяет значительно упростить разработку, 
так как уменьшает время на проверку гипотез и расходы на реальное оборудование, особенно в случае неудачных испытаний. Из-за этих факторов виртуальное моделирование 
в робототехнике приобрело широкое распространение и активно применяется, например, для разработки систем локализации и навигации беспилотного транспорта \cite{simulations}. 
Возросшее качество компьютерной графики к тому же позволило моделировать реалистичное окружение, что особенно важно при работе с системами технического зрения. 

Требованиями к виртуальной среде является возможность симулировать несколько широкоугольных камер и настраивать их параметры, 
легко интегрировать алгоритмы технического зрения и создать окружение, приближенное к тому, в котором может работать алгоритм в реальности. На данный момент исследователю доступен 
широкий выбор программного обеспечения, подходящего для этой задачи. В приложении А представлено сравнение имеющихся предложений по основным изложенным выше 
требованиям.


По результатам оценки собранные сведений принято решение проводить разработку в симуляторе Unity. Он позволяет подробно настраивать камеру и эмулировать fisheye-объектив, 
строить реалистичные сцены благодаря свободному импорту моделей, а при программировании в симуляторе можно использовать сторонние программы в виде динамически подключаемых библиотек. 
По функционалу так же подходит NVIDIA Isaac Sim, но от него пришлось отказаться из-за высоких системных требований и малой изученности продукта.     % TODO: звучит тупо

Разрабатываемое решение должно иметь возможность внедрения в ПО робота, % TODO: не совсем ПО всё-таки
поэтому должно реализовываться на одном из популярных и быстродейственных языков программирования. Учитывая необходимость интеграции с Unity и потребность использовать популярные 
библиотеки, выбран язык C++.  Другим важным фактором является библиотека обработки изображений. В качестве основы для программной части была выбрана библиотека OpenCV, являющаяся стандартом 
при разработке систем технического зрения. 

В ходе работы рассмотрены существующие системы стереозрения, в особенности использующие информацию
со сверхширокоугольных камер. Изучены модели, описывающие подобные камеры. Рассмотрено ПО для виртуального
 моделирования робототехнических комплексов и систем технического зрения. Выбрано ПО для моделировани и 
разработки системы стереозрения, использующей объективы "рыбий глаз". 

Разработан и описан алгоритм устранения искажений в области интереса и его математическая модель.   % пафосно, конечно.  Так ли хорошо она описана?
Описана возможная конфигурация системы стереозрения, рассмотрен принцип работы её фрагмента. Этот  фрагмент считать
был смоделирован в среде Unity. В процессе размещены и настроены камеры, подготовлены объекты для калибровки и 
дальнейших испытаний системы. Реализована передача изображений с виртуальных камер для обработки алгоритмами
компьютерного зрения. С помощью снимков из виртуальной модели исследовано качество устранения искажений и выполнено
 сравнение точности оценки глубина предлагаемой системы и варианта с обычными камерами. Результаты позволяют рассмотреть
применение системы с реальными камерами и указывают на возможные доработки.  % FIXME: "указывают на доработки" такое себе 

Дальнейшая работа будет сконцентрирована на оптимизации и повышении точности алгоритма устранения искажений,
разработке способа автоматической установки параметров системы при разных конфигурациях камер. После этого можно
будет приступить к испытаниям качества построения карты и локализации с применением описанной системы сначала
с виртуальным, а затем и с реальном роботом.
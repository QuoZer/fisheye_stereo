\subsection{Модель сверхширокоугольного объектива}

Сверхширокоугольные объективы имеют в своей основе сложную систему линз, схема которой представлена на рисунке \ref{pic:fyscheme}. 
Особенности этой системы позволяют достигать существенного угла обзора, но также являются причиной аберрации и характерных искажений 
изображения. 

\begin{figure}[H]
    \begin{center}
        \includegraphics[scale=0.5]{pics/fisheye_scheme.png}                                                                                            %TODO: перерисовать схему?
        \caption{Системы кругового обзора}
        \label{pic:fyscheme}
    \end{center}
\end{figure}
    
Перед использованием снимков с подобных камер необходимо избавиться от искажений. Для осуществления этого необходима модель камеры - 
набор уравнений, который позволяет найти проекцию точки в мировых координатах ${X_w, Y_w, Z_w}$ на точку в плоскости изображения ${u, v}$ и наоборот. 
Перспективная проекция, которая обычно используется для описания камер, не способна спроецировать широкоугольный снимок на кадр конечного размера. Поэтому при производстве
 fisheye-объективов опираются на другие виды проекций \ref{}. Но реальные линзы не всегда в точности следуют заданным моделям, по этой 
 причине для моделирования подобных искажений принято использовать многочлен вида

 \begin{equation}	% TODO: переписать уравнение 
	\begin{split}
        \delta r= k_1 r^3 + k_2 r^5 + k_3 r^7 + ... + k_n r^{n+2}
        \label{eqn:fisheye_distortion}
    \end{split}
\end{equation}

где $k_i$ - коэффициенты, описывающие внутренние параметры камеры. 

В настоящий момент есть несколько распространённых моделей, аппроксимирующих реальные искажения подобных объективов. Модель Канналы и 
Брандта \cite{opencv_model} реализована в OpenCV и описывает радиальные искажения через угол падения луча света на линзу, а не расстояние  
от центра изображения до места падения, как это делалось в более ранних моделях. Авторы посчитали, что для описания типичных искажений достаточно 
пяти членов полинома. Таким образом, указанную модель можно записать следующими уравнениями:

\begin{equation}	
        \theta = \arctan(\frac{r}{f}),
        \label{eqn:kannala_theta}
\end{equation}

\begin{equation}	
    \delta r = k_1\theta + k_2\theta^3 + k_3\theta^5 + k_4\theta^7 + k_5\theta^9,
    \label{eqn:kannala_r}
\end{equation}

\begin{equation}	
    \begin{pmatrix}u\\v\end{pmatrix} = \delta r(\theta)\begin{pmatrix}cos(\phi)\\sin(\phi)\end{pmatrix},
    %\delta r = k_1\theta + k_2\theta^3 + k_3\theta^5 + k_4\theta^7 + ... + k_n\theta^{n+1}
    \label{eqn:kannala_uv}
\end{equation}
где $\theta$ - угол падения луча, определяемый выбранным типом проекции; $\phi$ - угол между горизонтом 
и проекцией падающего луча на плоскость изображения; $r$ - расстояние от спроектированной точки до центра
изображения; $f$ - фокусное расстояние.

Также часто можно встретить модель Скарамуззы \cite{scaramuzza}, которая легла в основу Matlab Omnidirectional 
Camera Calibration Toolbox. Она связывает точки на изображении с соответствующей им точкой в координатах камеры 
следующим образом
\begin{equation}	
    \begin{pmatrix}X_c\\Y_c\\Z_c\end{pmatrix} = \lambda \begin{pmatrix}u\\v\\a_0 + a_2 r^2 + a_3 r^3 + a_4 r^4\end{pmatrix},
    %\delta r = k_1\theta + k_2\theta^3 + k_3\theta^5 + k_4\theta^7 + ... + k_n\theta^{n+1}
    \label{eqn:scaramuzza}
\end{equation}
где $a_0 ... a_4$ - коэффициенты, описывающие внутренние параметры камеры; $\lambda$ - размерный коэффициент.
% TODO: дописать



\subsection{Обзор существующих систем стереозрения, использующих сверхширокоугольные изображения}

Библиотеки машинного зрения предлагаю готовые к использованию классы и функции, позволяющие после калибровки
камер получать с помощью них карты глубины. Однако на практике эти методы весьма ограничены. Для стереосопоставления 
используются традиционные методы, приспособленные для классических камер с перспективной проекцией, что не позволяет 
использовать кадры с широкоугольных камер целиком. В результате у этих кадров после устранения искажений остаётся 
угол зрения порядка 90\deg. Кроме того, библиотечные функции не позволяют задать область интереса для каждой камеры, 
что ограничивает их область применения только для копланарного расположения камер. 

Исследователи предложили несколько реализаций стереозрения, опирающихся на снимки со сверхширокоугольных 
объективов.  

Есть неиросетевые методы ...
% Вообще, в этой работе и вся та же тема с поворотами есть 
Другой метод, предложенный в \cite{omni_stereo} позволяет создать кольцевую область стереозрения с вертикальным  %Метод с разворотом камер на 180  ...
полем зрения 65\deg. Для этого используются две 245\deg камеры, закреплённые на противоположных концах жёсткого стержня.  
Это позволяет достигнуть панорамного обзора глубины, но доступна такая схема расположения камер только летательным аппаратам.  

Были предложены копланарные методы, но до реализации не дошло ...
 
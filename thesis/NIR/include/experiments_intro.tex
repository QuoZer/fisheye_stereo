Описанная в предыдущей главе виртуальная модель системы стереозрения позволяет проводить 
с ней испытания для оценки работоспособности и сравнения с аналогами в контролируемой среде.            % FIXME: как же плохо

Результатом работы алгоритма стереозрения является карта глубины воспринимаемого пространства. Подобные карты могут
использоваться, например, алгоритмами навигации и локализации для построения карты окружения робота. Точность работы 
этих алгоритмов зависит от того, насколько точно карта глубины передаёт реальную информацию о форме объектов. Оценить 
эту характеристику для отдельной системы стереозрения проблематично, так как она зависит от множества факторов.         % TODO: дописать факторы или сослаться на статью
Поэтому оценка качества работы системы проведена в сравнении с виртуальной моделью традиционной стереопары.

Виртуальная сцена позволяет разместить сразу несколько объектов в одной точке пространства, таким образом возможно
в существующую модель в Unity добавить ещё 2 камеры, совпадающие по параметрам и положению с виртуальными камерами на 
рисунке \ref{pic:2cam_scheme}. Разрешение этих камер выбрано исходя из размеров проекции $\nu_i$ области интереса на широкоугольном
снимке. Для камер "рыбий глаз" с разрешением $1080*1080$ пикселей она составляет 287482 пикселей, что аналогично камере 
с разрешением $540*540$.  В результате получена эталонная стереопара для сравнения. % TODO: чувствуется какая-то недосказанность

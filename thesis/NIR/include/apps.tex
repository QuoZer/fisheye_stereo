\appsection{Приложение А} \hypertarget{app-a}{\label{app-a}}

\centering{Сравнение ПО для симуляции}


    \begin{table}[h!]             % TODO: Сыровато. Надо дополнить. Проверить ГОСТовость подписи  FIXME: Таблица не умещается, но вращать не хочется. 
        
        %\caption{Сравнение ПО для симуляции }
        \label{tab:sims}
        \rotatebox{90} {\begin{tabular}{|l|l|l|l|l|}
        \hline
        \textbf{Название} & \textbf{\begin{tabular}[c]{@{}l@{}}Симуляция \\ fisheye-камер\end{tabular}} & \textbf{\begin{tabular}[c]{@{}l@{}}Реалистичное \\ моделирование\end{tabular}} & \textbf{Интеграция кода}                                            & \textbf{Доступность}  \\ \hline
        Gazebo            & Возможна                                                                    & Возможно                                                                     & \begin{tabular}[c]{@{}l@{}}Возможна \\ посредством ROS\end{tabular} & Бесплатно               \\ \hline
        RoboDK            & Нет                                                                         & Затруднено                                                                     & Нет                                                                 & От 145€                \\ \hline
        Webots            & Затруднена                                                                  & Возможно                                                                       & Возможна                                                            & Бесплатно              \\ \hline
        CoppeliaSim       & Затруднена                                                                  & Затруднено                                                                     & Возможна                                                            & Бесплатно              \\ \hline
        NVIDIA Isaac Sim  & Возможна                                                                    & Возможно                                                                       & Возможна                                                            & Бесплатно              \\ \hline
        CARLA             & Затруднена                                                                  & Возможно                                                                       & Возможна                                                            & Бесплатно              \\ \hline
        Unity             & Возможна                                                                    & Возможно                                                                       & Возможна                                                            & Бесплатно              \\ \hline
        \end{tabular}}

    \end{table}
\clearpage


% \appsection{Приложение Б} \hypertarget{app-b}{\label{app-b}}

% \centering{Программный код файла FisheyeDewarper.cpp}
% \begin{english}
% \lstinputlisting[language=C++, numbers=left]{code/FisheyeDewarper.cpp}
% \end{english}

% \clearpage

% \appsection{Приложение В} \hypertarget{app-c}{\label{app-c}}

% \centering{Программный код файла unity\_plugin.cpp}
% \begin{english}
% \lstinputlisting[language=C++, numbers=left]{code/unity_plugin.cpp}
% \end{english}

% \clearpage

% \appsection{Приложение Г} \hypertarget{app-d}{\label{app-d}}

% \centering{Программный код файла Connector.cs}
% \begin{english}
% \lstinputlisting[language=C, numbers=left]{code/Connector.cs}
% \end{english}

% \clearpage

% \appsection{Приложение Д} \hypertarget{app:matlab}{\label{app:matlab}}

% \centering{Программный код файла DisparityPlayground.m}
% \begin{english}
% \lstinputlisting[language=Matlab, numbers=left]{code/DisparityPlayground.m}
% \end{english}

% \clearpage
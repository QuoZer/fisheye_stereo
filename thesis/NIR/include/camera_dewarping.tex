\subsection{Передача изображений с виртуальных камер для обработки}
Unity в качестве основного языка программирования использует C# и не позволяет использовать 
выбранную библиотеку технического зрения OpenCV напрямую. Однако как уже упоминалось в секции 
\ref{sec:software}, данная среда позволяет интегрировать плагины, в том числе написанные на 
других языках программирования, в виде динамически подключаемых библиотек (DLL). Именно этот 
подход был использован для передачи изображений - была разработана библиотека с функциями для 
работы с изображениями, которая компилировалась отдельно и затем импортировалась в Unity. Далее 
специальный скрипт в сцене переводил каждое обновление кадра изображения с виртуальных камер в 
понятный для библиотеки формат и вызывал      % FIXME: "понятный для библиотеки" - звучит ненаучно
нужные в текущий момент функции.    

Для реализации описанного принципа и упрощения разработки в библиотеке были реализованы следующие функции:
\begin{itemize}
    \item initialize - выделяет память под нужные структуры и создаёт окна для отображения будущих изображений.
    \item getImages - передаёт изображения в библиотеку и производит обработку изображений в 
    соответствии с аргументами\dots
    \item takeScreenshot - производит ту же обработку входного изображения, что и предыдущая функция,
     но результат сохраняет в файл. 
    \item ...    
\end{itemize}
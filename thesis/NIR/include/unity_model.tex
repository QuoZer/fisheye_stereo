\subsection{Виртуальное моделирование системы}

Описанная система была смоделирована в среде Unity, её внешний вид представлен на рисунке \ref{pic:unity_model}. Для 
моделирования камеры "рыбий глаз" использовался ад-дон Dome Projector из магазина Unity Asset Store.    % TODO: проверить все названия
Он позволяет моделировать сверхширокоугольные объективы с разным углом зрения в эквидистантной проекции \cite{}. % TODO: тут бы сослаться на их работку какую-нибудь
При этом с точки зрения других искажений, не относящихся к моделированию ___, снимки с этой камеры получаются идеальными. % FIXME: моделированию чего? 
Окно настроек камеры представлено на рисунке \ref{pic:camera_settings}.
\begin{figure}[H]
    \begin{center}
        \includegraphics[scale=0.5]{pics/camera_settings.png}                                                                                            %TODO: перерисовать схему?
        \caption{Окно настроек виртуальной камеры в Unity}
        \label{pic:camera_settings}
    \end{center}
\end{figure}
Для виртуальной камеры доступны настройки угла зрения и виньетки по краям изображения,  % FIXME: виньетки -> абберации ??
а также различных параметров рендеринга, влияющих на качество получаемого изображения. 
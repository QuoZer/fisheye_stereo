\documentclass[a4paper,14pt]{extarticle} % 14й шрифт
%%% Преамбула %%%

\usepackage{fontspec} % XeTeX
\usepackage{xunicode} % Unicode для XeTeX
\usepackage{xltxtra}  % Верхние и нижние индексы
\usepackage{pdfpages} % Вставка PDF
\usepackage{bm}

\usepackage{listings} % Оформление исходного кода
\lstset{
    basicstyle=\small\ttfamily, % Размер и тип шрифта
    breaklines=true, % Перенос строк
    tabsize=2, % Размер табуляции
    literate={--}{{-{}-}}2 % Корректно отображать двойной дефис
}
% Русский язык
\usepackage{polyglossia}
\setdefaultlanguage{russian}
\setotherlanguage{english}

% Шрифты, xelatex
\defaultfontfeatures{Ligatures=TeX}
\setmainfont[Ligatures=TeX]{Times New Roman}
\newfontfamily\cyrillicfont{Times New Roman}[Script=Cyrillic]
%\setsansfont{Liberation Sans} % Альтернатива нью роману, если его нет
\setmonofont{FreeMono} % Моноширинный шрифт для оформления кода
\newfontfamily{\cyrillicfonttt}{FreeMono}


\usepackage{amssymb,amsfonts,amsmath} % Математика
\numberwithin{equation}{section} % Формула вида секция.номер

\usepackage{enumerate} % Тонкая настройка списков
\usepackage{indentfirst} % Красная строка после заголовка
\usepackage{float} % Расширенное управление плавающими объектами
\usepackage{multirow} % Сложные таблицы
\usepackage{adjustbox}
\usepackage{lscape}
\usepackage{rotating}

\usepackage{pdfcomment}     % тудущки и комменты
\pdfcommentsetup{final} % set to 'final' to prevent all comments from showing up   draft
\pdfcommentsetup{color={1.0 1.0 0.0}, open=true}

% Пути к каталогам с изображениями
\usepackage{graphicx} % Вставка картинок и дополнений
\graphicspath{{pics/}{images/userguide/}{images/testing/}{images/infrastructure/}{extra/}{extra/drafts/}}

% Формат подрисуночных записей
\usepackage{chngcntr} 
\counterwithin{figure}{section}

% Гиперссылки
\usepackage{hyperref}
\hypersetup{
    colorlinks, urlcolor={black}, % Все ссылки черного цвета, кликабельные
    linkcolor={black}, citecolor={black}, filecolor={black},
    pdfauthor={Author},
    pdftitle={Title}
}
% Оформление библиографии и подрисуночных записей через точку
\makeatletter
\renewcommand*{\@biblabel}[1]{\hfill#1.}
\renewcommand*\l@section{\@dottedtocline{1}{1em}{1em}}
\renewcommand{\thefigure}{\thesection.\arabic{figure}} % Формат рисунка секция.номер
\renewcommand{\thetable}{\thesection.\arabic{table}} % Формат таблицы секция.номер
\def\redeflsection{\def\l@section{\@dottedtocline{1}{0em}{10em}}}
\makeatother

\renewcommand{\baselinestretch}{1.4} % Полуторный межстрочный интервал
\parindent 1.27cm % Абзацный отступ

\sloppy             % Избавляемся от переполнений
\hyphenpenalty=1000 % Частота переносов
\clubpenalty=10000  % Запрещаем разрыв страницы после первой строки абзаца
\widowpenalty=10000 % Запрещаем разрыв страницы после последней строки абзаца

% Отступы у страниц
\usepackage{geometry}
\geometry{left=3cm}
\geometry{right=1.5cm}
\geometry{top=2cm}
\geometry{bottom=2cm}

% Списки
\usepackage{enumitem}
\setlist[enumerate,itemize]{leftmargin=0mm,itemindent = 20mm} % Отступы в списках

\makeatletter
    \AddEnumerateCounter{\asbuk}{\@asbuk}{м)}
\makeatother
\setlist{nolistsep} % Нет отступов между пунктами списка
\renewcommand{\labelitemi}{--} % Маркет списка --
\renewcommand{\labelenumi}{\asbuk{enumi})} % Список второго уровня
\renewcommand{\labelenumii}{\arabic{enumii})} % Список третьего уровня
                    % Глубина оглавления, до subsubsection
\usepackage{scrextend}


% Нумерация страниц по центру снизу
\usepackage{fancyhdr}
\pagestyle{plain}
\fancyhf{}
\fancyhead[R]{\textrm{\thepage}}
\fancyheadoffset{0mm}
\fancyfootoffset{0mm}
\setlength{\headheight}{17pt}
\renewcommand{\headrulewidth}{0pt}
\renewcommand{\footrulewidth}{0pt}
\fancypagestyle{plain}{ 
    \fancyhf{}
    \rhead{\thepage}
}
% Содержание
\usepackage{tocloft}
\renewcommand{\cftaftertoctitle}{
    \thispagestyle{plain}}     % Убирает номер страницы
\renewcommand{\cfttoctitlefont}{\hspace{0.38\textwidth}\MakeTextUppercase} % СОДЕРЖАНИЕ
\renewcommand{\cftsecfont}{\hspace{0pt}}            % Имена секций в содержании не жирным шрифтом
\renewcommand\cftsecleader{\cftdotfill{\cftdotsep}} % Точки для секций в содержании
\renewcommand\cftsecpagefont{\mdseries}             % Номера страниц не жирные
\setcounter{tocdepth}{3}        

% GOST - biblatex
\usepackage[% 
bibstyle=gost-numeric, %подключение одного из четырех главных стилей biblatex-gost 
sorting=none, %тип сортировки в библиографии
doi=false,  
]{biblatex}
\addbibresource{refs.bib}

% Формат подрисуночных надписей
\RequirePackage{caption}
\DeclareCaptionLabelSeparator{defffis}{ -- } % Разделитель
\captionsetup[figure]{justification=centering, labelsep=defffis, format=plain, belowskip=-15pt} % Подпись рисунка по центру
\captionsetup[table]{justification=raggedright, labelsep=defffis, format=plain, singlelinecheck=false} % Подпись таблицы слева
\addto\captionsrussian{\renewcommand{\figurename}{Рисунок}} % Имя фигуры

% Пользовательские функции
\newcommand{\addimg}[4]{ % Добавление одного рисунка
    \begin{figure}
        \centering
        \includegraphics[width=#2\linewidth]{#1}
        \caption{#3} \label{#4}
    \end{figure}
}
\newcommand{\addimghere}[4]{ % Добавить рисунок непосредственно в это место
    \begin{figure}[H]
        \centering
        \includegraphics[width=#2\linewidth]{#1}
        \caption{#3} \label{#4}
    \end{figure}
}
\newcommand{\addtwoimghere}[6]{ % Вставка двух рисунков
    \begin{figure}[H]
        \centering
        \includegraphics[width=#2\linewidth]{#1}
        \hspace{1cm}
        \includegraphics[width=#4\linewidth]{#3}
        \caption{#5} \label{#6}
    \end{figure}
}
\newcommand{\addimgapp}[2]{ % Это костыль для приложения Б
    \begin{figure}[H]
        \centering
        \includegraphics[width=1\linewidth]{#1}
        \caption*{#2}
    \end{figure}
}

% Заголовки секций в оглавлении в верхнем регистре
\usepackage{textcase}
\makeatletter
\let\oldcontentsline\contentsline
\def\contentsline#1#2{
    \expandafter\ifx\csname l@#1\endcsname\l@section
        \expandafter\@firstoftwo
    \else
        \expandafter\@secondoftwo
    \fi
    {\oldcontentsline{#1}{#2}}          % {\oldcontentsline{#1}{\MakeTextUppercase{#2}}} 
    {\oldcontentsline{#1}{#2}}
}
\makeatother

% Оформление заголовков
\usepackage[compact,explicit]{titlesec}
\titleformat{\section}{\large\bfseries}{}{0mm}{\centering{\thesection\quad\textbf{#1}} }  % \titleformat{\section}{}{}{12.5mm}{\centering{\thesection\quad\MakeTextUppercase{#1}}\vspace{1.5em}}
\titleformat{\subsection}[block]{\bfseries}{}{12.5mm}{\thesubsection\quad\textbf{#1}}          %
\titleformat{\subsubsection}[block]{\vspace{1em}\normalsize}{}{12.5mm}{\thesubsubsection\quad#1\vspace{1em}}
\titleformat{\paragraph}[block]{\normalsize}{}{12.5mm}{\MakeTextUppercase{#1}}


% Секции без номеров (введение, заключение...), вместо section*{}
\newcommand{\anonsection}[1]{
    \phantomsection % Корректный переход по ссылкам в содержании
    \paragraph{\centerline{{#1}}\vspace{0em}}
    \addcontentsline{toc}{section}{\uppercase{#1}}
}

% Секции для приложений
\newcommand{\appsection}[1]{
    \phantomsection
    \paragraph{\centerline{{#1}}}
    \addcontentsline{toc}{section}{\uppercase{#1}}
}

% Библиография: отступы и межстрочный интервал
% \makeatletter
% \defbibenvironment{thebibliography}
% {
%     \list{\@biblabel{\@arabic\c@enumiv}}
%         {\settowidth\labelwidth{\@biblabel{#1}}
%         \leftmargin\labelsep
%         \itemindent 16.7mm
%         \@openbib@code
%         \usecounter{enumiv}
%         \let\p@enumiv\@empty
%         \renewcommand\theenumiv{\@arabic\c@enumiv}}
%         }
%     \setlength{\itemsep}{0pt}
%     }
% \makeatother

\setcounter{page}{2} % Начало нумерации страниц % Подключаем преамбулу

%%% Начало документа
\begin{document}

\includepdf{include/titlepage}
%\includepdf{pz} % Пояснительная записка
%\includepdf[pages={1,2}]{task} % Задание на диплом 
\include{include/referat.tex}

\tableofcontents % Содержание 
\clearpage

\anonsection{введение}

Получение трёхмерной структуры пространства по стереоснимкам - это задача, первые решения которой
были получены десятилетия назад. Ранние работы фокусировались в основном на способах поиска соответствий
и геометрических основах, лежащих в основе процесса. Существенный объём научной работы продолжает
 производиться в области стереозрения и по сей день. Был достигнут заметный прогресс в повышении точности 
 результатов и понижении вычислительных мощностей, требуемых для их достижения, однако эти области остаются 
 фокусом исследований. 

Улучшение точности и производительности алгоритмов является нетривиальной задачей. На точность 
полученных результатов оказывает влияние нехватка информации, вызванная заслонением объектов, наличием наклонных
плоскостей и другими факторами, влияющими на сложность восстановления трёхмерных объектов. Разрешение
сенсоров также растёт с каждым годом, увеличивая вычислительную сложность поиска соответствий на кадрах с 
каждой камеры. Таким образом, исследователи в области стереозрения пытаются найти компромисс между этими
 двумя характеристиками. Однако для каждого конкретного алгоритма этот компромисс может быть смещён в 
 ту или иную сторону. \cite{Hartley2004}
\section{Аналитический обзор систем стереозрения}

\subsection{Стереозрение}
\label{stereovision}
Система стереозрения состоит из двух камер, наблюдающих сцену с разных точек, как изображено на рисунке \ref{pic:epipol} \cite{Hartley2004}. 
Фундаментальная основа принципа заключается в предположении, что каждой точке в пространстве соответствует уникальная пара пикселей на снимках с двух камер.  

При этом к камерам предъявляются некоторые требования \cite{rusoverview}:   % не уверен, что это надо цитировать
\begin{itemize}
	\item Камеры откалиброваны. Это значит, что известны внутренние (оптические) и внешние (расположение камер в пространстве) параметры камер. 
	\item Ректификация. Подразумевает выравнивание изображения с обеих камер по строкам.  % Мб подробнее расписать  
	\item Ламбертовость поверхностей. Означает независимость освещения наблюдаемых поверхностей от угла зрения. 
\end{itemize}

Таким образом, соблюдение указанных выше требований позволяет использовать следующий геометрический принцип. При наличии двух камеры, как изображено 
на рисунке \ref{pic:epipol}, где $C$ — центр первой камеры, $C'$ — центр второй камеры, точка пространства $X$  
проецируется в $x$ на плоскость изображения левой камеры и в $x'$ на плоскость изображения правой камеры. Прообразом точки $x$ на изображении левой 
камеры является луч $xX$. Этот луч проецируется на плоскость второй камеры в прямую $l'$, называемую эпиполярной линией. Образ точки $X$ на плоскости 
изображения второй камеры обязательно лежит на эпиполярной линии $l'$.

\addimghere{epipolar geometry}{0.5}{Эпиполярная геометрия}{pic:epipol}

В результате каждой точке $x$ на изображении левой камеры соответствует эпиполярная линия $l'$ на изображении правой камеры. При этом соответствие для $x$ на 
изображении правой камеры может лежать только на соответствующей эпиполярной линии. Аналогично, каждой точке $x'$ на правом изображении соответствует 
эпиполярная линия $l$ на левом. 

% TODO: стересопоставление

Далее с помощью точек $x$ и $x'$ возможно посчитать смещения каждого пикселя одного изображения относительно другого, что позволяет построить карту смещений. 
Очевидно, что смещения будут подсчитаны только для точек, видимых обеими камерами. Карта смещений же приводится далее либо к облаку точек, либо к карте глубины. 
Стереосистемы, реализующие этот принцип, называют пассивными. Они являются самыми простыми и часто используются, так как для их изготовления достаточно 
двух закреплённых камер. Однако пассивные системы опираются целиком на видимый свет, что ограничивает их применимость в условиях с малой освещённостью. %\cite{find:passive_perfomance}. 

Одну из камер можно заменить источником  света, освещающим одну или несколько точек поверхности световым лучом или специальным шаблоном освещения.  % TODO: цитирование страницы  529
Так как структура шаблона и направление его  лучей заранее известны, камера может проводить оценку формы объектов в кадре по искажениям шаблона\cite{shapiro}. 
Структурная подсветка тоже является распространённым методом, реализованным во многих коммерческих сенсорах,  благодаря совмещению высокой производительности 
и  низкой цены \cite{struct_light}.  

Совмещение пассивного стереозрения и структурированной подсветки позволяет улучшить качество стереосопоставления, особенно для поверхностей со слабо выраженной текстурой
и при низкой освещённости. Однако инфракрасный узор подсветки весьма ограничен в дальности и не виден при сильном солнечном освещении, что затрудняет использование этих    % FIXME: криво ппц
сенсоров на открытых пространствах. Системы с подобной технологий называют активными \cite{kinect_perf}. 

\addimghere{stereo_types}{1.0}{Виды организации стереосистем: а) пассивная стереосистема; б) активная стереосистема; в) подсветка структирированным светом }{pic:epipol}
\pdfmargincomment{просится какой-то вывод}
Описанные подходы хорошо работают для обычных камер с незначительными искажениями, где выполняется эпиполярная геометрия. \pdfmargincomment{и сюда сразу модель обскуры или потом оставить?}
В случае  же камер со сверхшироким углом обзора в изображение вносятся существенные искажения, которые затрудняют поиск соответствий. На 
рисунке \ref{pic:fy_epipol} вручную сопоставлены одни и те же точки на фрагментах двух изображений. Видно, что пары признаков больше не лежат
на одной горизонтальной прямой.  Учитывая распространённость таких  камер в робототехнике, задача реализации систем стереозрения на их основе 
является актуальной.    % Вообще какое-то введение опять, действительно ли распространены

\addimghere{not_epipolar}{0.8}{Соответствия на снимках с fisheye-камер}{pic:fy_epipol}   % TODO: найти/нарисовать

Исследователи предложили несколько реализаций систем стереозрения, опирающихся на снимки со сверхширокоугольных камер. 
Например, с помощью  пары таких камер реализована  кольцевая область стереозрения с вертикальным  %Метод с разворотом камер на 180  ...
полем зрения $65^\circ$\cite{omni_stereo}. Для этого две $245^\circ$ камеры закреплены на противоположных концах жёсткого стержня.  
Это позволяет достигнуть панорамного обзора глубины с качеством, достаточным для осуществления автономной навигации и
локализации \cite{omni_copter}, но конструктивно реализуема такая схема расположения камер только для летательных аппаратов.  

Рохас и Оиши \cite{direct_neuro_stereo} решили отказаться от типичных для стереозрения этапов устранения искажений и ректификации % FIXME: пояснить ректификацию? 
и извлекать информацию о глубине напрямую по двум снимкам fisheye-камер. Для производства карт глубины используется 
свёрточная неиронная сеть, что требует существенных вычислительных мощностей - для достижения производительности в реальном 
времени разработчикам понадобилось использовать компьютер с ЦПУ i7-4770 и ГПУ NVIDIA GTX 1080Ti. В мобильном автономном роботе
аналогичный по производительности вычислитель разместить может быть проблематично. Кроме того, метод описан лишь для случая, когда
 обе камеры направлены в одном направлении. 

Были предложены ортогональные системы, использующие большие площади перекрытия полей зрения объективов "рыбий глаз" для 
стереозрения. % FIXME: фиксить формулировку, некрасиво
Чжан и другие разработали особую систему стереозрения (the special stereo vision system), использующую модуль из четырёх камер с углом зрения
$185^\circ \\times 185^\circ$. Размещёны камеры в одной плоскости под углом в $90^\circ$, как изображено на рисунке \ref{pic:ssvs} \cite{zhang_system}.
Авторы отразили в работе калибровку разработанной ими системы и устранение искажений, но не проанализировали точность метода. Кроме того,
в работе не отражены  результаты  оценки глубины. 

\addimghere{ssvs}{0.7}{Модуль камер}{pic:ssvs}

% Распространённые библиотеки машинного зрения (OpenCV, MATLAB CV Toolbox) предлагаю готовые к использованию классы и функции, позволяющие после калибровки
% камер получать с помощью них карты глубины. Однако на практике эти методы весьма ограничены. Для стереосопоставления 
% используются традиционные методы, приспособленные для классических камер с перспективной проекцией, что не позволяет 
% использовать кадры с широкоугольных камер целиком. В результате у этих кадров после устранения искажений остаётся 
% угол зрения не более $120^\circ$. Кроме того, библиотечные функции не позволяют задать область интереса для каждой камеры, 
% что ограничивает их область применения только для копланарного расположения сенсоров.              % Обзор систем
\subsection{Модели сверхширокоугольной камеры}
\label{camera_model}
Сложности, возникающие при использовании существующих алгоритмов стереозрения  в применении к сверхширокоугольным камерам, связаны с
 особенностями их оптической системы. Объективы  этих камер имеют в своей основе сложную систему линз, схема которой вместе с примером
 получаемого изображения представлена на рисунке \ref{pic:fyscheme}. Особенности этой системы позволяют достигать углов обзора свыше $180^\circ$,
  но также являются причиной аберрации и характерных искажений изображения. Реалистично моделировать ход лучей в подобных камерах 
нецелесообразно, поэтому исследователи прибегают к аппроксимациям, называемым моделями камер.   % FIXME: meh

\addtwoimghere{fisheye_scheme}{0.45}{fisheye_example}{0.45}{Схема хода лучей объектива "рыбий глаз" (слева), пример изображения (справа)\cite{fy_exmp}}{pic:fyscheme}

Как видно по рисунку \ref{pic:fy_geom}, модель проекции для камеры это функция, которая моделирует преобразование 
из точки трёхмерного пространства  в области зрения камеры ($P=[x_c, y_c, z_c]^T$) в точку на плоскости изображения ($p=[u, \nu]^T$). Единичная            % не совсем единичная 
полусфера $S$ с центром в точке $O_c$ описывает поле зрения. На ней также лежит точка $P_C$, являющаяся результатом обратной проекции.    %$\pi^{-1}_c({p})$
Угол $\theta$ является углом падения для рассматриваемой точки, а угол $\phi$ откладывается между положительным направлением оси $x$ и $O_{i}{p}$. 


Модели камер включают в себя описания нескольких типов искажений, накладываемых линзой, но в сверхширокоугольных объективах самыми существенными являются 
радиальные - искажения, проявляющиеся сильнее ближе к краям изображения. Поэтому далее в этой секции модели будут рассматриваться именно с точки зрения 
описания радиальных искажений. 

\addimghere{projection_geometry}{0.5}{Схема проекции точки трёхмерного пространства в точку на изображении}{pic:fy_geom}

Перспективная проекция, которая обычно используется в качестве модели ортоскопической камеры, не способна спроецировать широкоугольное пространство на кадр 
конечного размера. Поэтому при описании и разработке fisheye-объективов опираются на другие виды проекций  \cite{projections}: 
\begin{equation}
    \label{fy1}
r = 2 f tan(\theta/2),  
\end{equation}
\begin{equation}
    \label{fy2}
r = f \theta,
\end{equation}
\begin{equation}
    \label{fy3}
r = 2 f sin(\theta/2),
\end{equation}
\begin{equation}
    \label{fy4}
r = f sin(\theta),
\end{equation}

Но реальные линзы не всегда в точности следуют заданным моделям, к тому же отличия в используемых параметрах усложняют процесс калибровки камер. 
По этой причине радиальные искажения выгоднее аппроксимировать многочленами \cite{opencv_model}, например, вида
 \begin{equation}	
	\begin{split}
        \delta r= k_1 r^3 + k_2 r^5 + k_3 r^7 + ... + k_n r^{n+2},
        \label{eqn:fisheye_distortion}
    \end{split}
\end{equation}
где $\delta r$ - радиальное отклонение от идеальной проекции луча;  $k_i$ - коэффициенты, описывающие внутренние параметры камеры. 

В настоящий момент есть несколько распространённых моделей, аппроксимирующих реальные искажения подобных объективов. Модель Канналы и 
Брандта \cite{opencv_model} для линз с радиально симметричными искажениями реализована в OpenCV и выражает их через угол падения луча света на линзу, а не расстояние  \pdfmargincomment{https://stackoverflow.com/questions/31089265/what-are-the-main-references-to-the-fish-eye-camera-model-in-opencv3-0-0dev}
от центра изображения до места падения, как это делалось в более ранних моделях. Авторы посчитали, что для описания типичных искажений достаточно 
пяти членов полинома. Таким образом, указанную модель можно записать следующими уравнениями:

\begin{equation}	
    \delta r = k_1\theta + k_2\theta^3 + k_3\theta^5 + k_4\theta^7 + k_5\theta^9,
    \label{eqn:kannala_r}
\end{equation}

\begin{equation}	
    \begin{pmatrix}u\\v\end{pmatrix} = \delta r(\theta)\begin{pmatrix}cos(\phi)\\sin(\phi)\end{pmatrix},
    %\delta r = k_1\theta + k_2\theta^3 + k_3\theta^5 + k_4\theta^7 + ... + k_n\theta^{n+1}
    \label{eqn:kannala_uv}
\end{equation}
где $\theta$ - угол падения луча, определяемый выбранным типом проекции; $\phi$ - угол между горизонтом 
и проекцией падающего луча на плоскость изображения; $r = sqrt(x^2+y^2)$ - расстояние от спроектированной точки до центра
изображения; $f$ - фокусное расстояние.

Также большое распространение получила модель Скарамуззы \cite{scaramuzza}, которая легла в основу Matlab Omnidirectional 
Camera Calibration Toolbox. Она связывает точки на изображении с соответствующей им точкой в координатах камеры фьук
следующим образом
\begin{equation}	
    \begin{pmatrix}X_c\\Y_c\\Z_c\end{pmatrix} = \lambda \begin{pmatrix}u\\v\\a_0 + a_2 r^2 + a_3 r^3 + a_4 r^4\end{pmatrix},
    %\delta r = k_1\theta + k_2\theta^3 + k_3\theta^5 + k_4\theta^7 + ... + k_n\theta^{n+1}
    \label{eqn:scaramuzza}
\end{equation}
где $a_0 ... a_4$ - коэффициенты, описывающие внутренние параметры камеры; $\lambda$ - масштабный коэффициент.

В качестве модели fisheye-камеры в этой работе принята именно модель Скарамуззы, так как она обладает точностью на уровне 
других \cite{double_sphere} и удобным инструментом оценки внутренних параметров камеры.

% Уменьшить  число параметров калибровки (принять $a_1$) в этой модели позволяет наблюдение 

Существуют и менее распространённые модели, не использующие полиномы для описания искажений. Одной из них является
модель двух сфер \cite{double_sphere}. % TODO: перевести получше
Она положение пикселя, проектируя точку сначала на первичную сферу, потом на вторую сферу поменьше, сдвинутую на 
расстояние $\xi$, а затем в плоскость изображения камеры-обскуры, сдвинутой на $\frac{\alpha}{1-\alpha}$ относительно 
центра второй сферы. Модель проекции представлена на рисунке \ref{pic:ds_model}. Таким образом радиальные искажения 
можно описать всего двумя параметрами. Модель также интегрирована
во многие популярные программы для калибровки камер (Basalt, Kalibr).

\addimghere{double_sphere}{0.7}{Модель  двух сфер}{pic:ds_model}



               % Обзор моделей камер
\subsection{Обоснование выбора ПО}
\label{sec:software}
Разработку и первоначальные испытания алгоритма стереозрения целесообразно проводить в виртуальной среде. Это позволяет значительно упростить разработку, 
так как уменьшает время на проверку гипотез и расходы на реальное оборудование, особенно в случае неудачных испытаний. Из-за этих факторов виртуальное моделирование 
в робототехнике приобрело широкое распространение и активно применяется, например, для разработки систем локализации и навигации беспилотного транспорта \cite{simulations}. 
Возросшее качество компьютерной графики к тому же позволило моделировать реалистичное окружение, что особенно важно при работе с системами технического зрения. 

Требованиями к виртуальной среде является возможность симулировать несколько широкоугольных камер и настраивать их параметры, 
легко интегрировать алгоритмы технического зрения и создать окружение, приближенное к тому, в котором может работать алгоритм в реальности. На данный момент исследователю доступен 
широкий выбор программного обеспечения, подходящего для этой задачи. В приложении А представлено сравнение имеющихся предложений по основным изложенным выше 
требованиям.


По результатам оценки собранные сведений принято решение проводить разработку в симуляторе Unity. Он позволяет подробно настраивать камеру и эмулировать fisheye-объектив, 
строить реалистичные сцены благодаря свободному импорту моделей, а при программировании в симуляторе можно использовать сторонние программы в виде динамически подключаемых библиотек. 
По функционалу так же подходит NVIDIA Isaac Sim, но от него пришлось отказаться из-за высоких системных требований и малой изученности продукта.     % TODO: звучит тупо

Разрабатываемое решение должно иметь возможность внедрения в ПО робота, % TODO: не совсем ПО всё-таки
поэтому должно реализовываться на одном из популярных и быстродейственных языков программирования. Учитывая необходимость интеграции с Unity и потребность использовать популярные 
библиотеки, выбран язык C++.  Другим важным фактором является библиотека обработки изображений. В качестве основы для программной части была выбрана библиотека OpenCV, являющаяся стандартом 
при разработке систем технического зрения.        % Выбор ПО TODO: убрать? 
\subsection{Выводы по главе}

Обзор современных моделей сверхширокоугольных камер позволил выбрать 
наиболее точную и удобную для калибровки. Был осуществлён обзор существующих
систем стереозрения, применяющих камеры типа "рыбий глаз" с целью ознакомления 
с мировым опытом. Было принято решение разрабатывать и тестировать предлагаемую 
систему стереозрения с применением виртуального  моделирования. 

Для моделирования выбрана среда разработки Unity. Для обработки изображений
с камер выбрана библиотека OpenCV для языка программирования C++.             % Выводы по разделу 

\newpage
\section{Система стереозрения}
\subsection{Описание системы стереозрения}

Предлагаемая система стереозрения состоит из двух камер с объективами типа "рыбий глаз" $\geqslant180^\circ$,
расположенных ортогонально так, что  две камеры имеют область пересечения полей зрения. В пространстве, наблюдаемом 
сразу  двумя камерами возможна триангуляция и получение информации об объёме.  % FIXME: Всё описание просто бггг

Рассмотрим организацию системы на примере робота-доставщика "Ровер R3"  компании Яндекс, который имеет на борту 4 сверхширокоугольные камеры, 
размещённые спереди, сзади и по бортам корпуса \cite{yandex_rover}, что соответствует описанию системы. 
На рисунке \ref{pic:4cam_system}, а показано реальное положение камер  робота и их зон перекрытия, в которых может 
 обеспечивается получение информации о глубине при использовании описываемой системы. Рисунок \ref{pic:4cam_system}, б
демонстрирует эквивалентную по горизонтальному  покрытию схема при использовании обычных камер 
 с углом обзора $90^\circ$. 
 
%\addtwoimghere{Group 1}{0.4}{Group 2}{0.4}{Сравнение систем стереозрения}{pic:4cam_system}  % TODO: разобраться с масштабом
\addimghere{group12}{0.7}{Геометрическая модель бинокулярной системы стереозрения}{pic:4cam_system}

Как можно заметить, системе на основе обычных камер нужно в два раза больше сенсоров, чтобы достичь той же зоны покрытия 
по горизонтали.  Кроме того, традиционная система имеет меньшую зону покрытия по вертикали и не обеспечивает полный 
панорамный обзор. Все эти факторы делают систему стереозрения на основе ортогонально расположенных сверхширокоугольных камер 
более предпочтительной для применения в робототехнике.  % FIXME: уточнить. не во всей не всегда. сскорее выгодной, но  слово не очень

Применение существующих алгоритмов стереосопоставления предполагает наличие стереопары, удовлетворяющей условиям, описанным в секции \ref{stereovision}.
 Такую стереопару можно получить, введя в систему  для каждой сверхширокоугольной камеры виртуальную(ые) камеру-обскуру и направив 
её в сторону пересечения полей зрения, как если бы это была пара обычных камер. Процесс формирования виртуальной камеры-обскуры и 
алгоритм устранения искажений более подробно описаны в секции \ref{dewarping}.

Далее для упрощения рассмотрения системы будет считаться, что оптические оси всех камер находятся в одной плоскости, 
а на ориентацию виртуальных камер влияет только угол $ \beta $ поворота в этой плоскости. 

На рисунке \ref{pic:2cam_scheme} изображён простейший вариант системы с двумя камерами под углом $90^\circ$, который можно  считать частью
схемы, представленной на рисунке \ref{pic:4cam_system}, или модуля, изображённого  на рисунке \ref{pic:ssvs}. 

\addimghere{sample_simple2cam}{0.7}{Геометрическая модель бинокулярной системы стереозрения}{pic:2cam_scheme} %TODO: перерисовать схему?

  Здесь область пересечения полей зрения камер $C_0$ и $C_1$ обозначенная красным.
Эта область эквивалентна области пересечения полей зрения двух камер с полями зрения $90^\circ$ (обозначены
оранжевым), повёрнутых на $\beta_0 = \beta_1  = 45^\circ$ в сторону области интереса. Тогда $B$ - база стереопары.
Примеры изображений, полученных в такой конфигурации, приведён на рисунке \ref{pic:dewarped_exmples}. 

\addimghere{4pic_example}{0.7}{Пример исходных изображений с отмеченной областью интереса и снимков виртуальной стереопары}{pic:dewarped_exmples}

      % Описание системы: Какие этапы, элементы и тд
\subsection{Виртуальное моделирование системы}
Описанная система была смоделирована в среде Unity, её внешний вид представлен на рисунке \ref{pic:unity_model}. Мир 
Unity предназначен для базовой проверки работоспособности испытываемого принципа, поэтому не содержит подробной модели
какого-либо робота. В нём присутствуют: плоскость земли, кронштейн, на котором сверхширокоугольные камеры закреплены под 
углом $90^\circ$, подвижный калибровочный узор и объекты-цели, предназначенные для оценки расстояния. В силу особенностей 
работы многих алгоритмов стереозрения эти объекты должны сильнотекстурированы \cite{disparity_review}. 

\addimghere{unity_view}{0.7}{Внешний вид сцены в Unity}{pic:unity_model} 

В Unity создание сцены происходит с использованием встроенных примитивов, импортированных файлов моделей популярных форматов
или моделей из магазина ассетов, предлагающего обширную библиотеку объектов и текстур, повторяющих различные реальные объекты. В данной 
работе использовались как и примитивы для создания простых объектов, так и модели из магазина для имитации препятствий и окружения.

Для моделирования камеры "рыбий глаз" использовался аддон Dome Tools из магазина Unity Asset Store.    
Он позволяет моделировать сверхширокоугольные объективы с разным углом зрения в эквидистантной проекции \cite{dome_tools}. 
При этом с точки зрения других искажений, не относящихся к моделированию правильной проекции, снимки с этой камеры получаются идеальными. % FIXME: моделированию чего? 
Окно настроек камеры представлено на рисунке \ref{pic:camera_settings}.

\addimghere{camera_settings}{0.5}{Окно настроек виртуальной камеры в Unity}{pic:camera_settings} 

Для виртуальной камеры доступны настройки угла зрения и виньетки по краям изображения,  % FIXME: виньетки -> абберации ??
а также различных параметров рендеринга, влияющих на уровень  детализации получаемого изображения.       % Моделирование системы 
\subsection{Передача изображений с виртуальных камер для обработки}

Unity в качестве основного языка программирования использует C\# и не позволяет импортировать 
выбранную библиотеку технического зрения OpenCV напрямую. Однако как уже упоминалось в секции 
\ref{sec:software}, данная среда позволяет интегрировать плагины, в том числе написанные на 
других языках программирования, в виде динамически подключаемых библиотек (DLL). Именно этот 
подход был использован для передачи изображений - была разработана библиотека с функциями для 
обработки снимков, которая компилировалась отдельно и затем импортировалась в Unity. 

Для реализации описанного в этой работе принципа и упрощения разработки в библиотеке были реализованы 
следующие функции:
\begin{itemize}
    \item initialize - выделяет память под нужные структуры, заполняет таблицу поиска и создаёт окна 
    для отображения будущих изображений.
    \item getImages - передаёт изображения в библиотеку и производит обработку изображений в 
    соответствии с аргументами. 
    \item takeScreenshot - производит ту же обработку входного изображения, что и предыдущая функция,
     но результат сохраняет в файл. 
    \item processImage - передаёт обратно в скрипт уже обработанные изображения для удобного отображения
    в интерфейсе Unity. 
    \item terminate - высвобождает память по завершении работы симуляции.      
\end{itemize}

Обработка, упомянутая в пунктах с функциями getImages и takeScreenshot, заключается в конвертации 
цветового пространства изображений (из RGBA, используемого в Unity, в BGR, используемый в OpenCV), 
горизонтальном зеркальном отображении (из-за разного положения точки отсчёта координат пикселей)
и, наконец, устранениb искажений в области интереса. Так как для наилучшей работы стереосопоставления
снимки должны быть синхронизированы (то есть получены камерами в один момент времени), обработка        % FIXME: ну не совсем одновременно и мб лучше это описать с другого ключа - типа функция может обрабатывать n снимков?
производится сразу для двух изображений. 

Также написан специальный скрипт, который управляет процессом взаимодействия симуляции и библиотеки. 
Эта программа имеет окно настроек, которое позволяет задать
используемые камеры, элементы управления параметрами  и выбрать область интереса с помощью задания углов.
 Внешний вид этого окна представлен на рисунке \ref{pic:connec_inter}.

\addimghere{connector_settings}{0.7}{Внешний вид окна настроек скрипта}{pic:connec_inter}

Сразу после запуска симуляции происходит запуск дополнительных вычислительных потоков для обработки 
кадров с каждой пары камер и передача настроек скриптом в библиотеку для построения таблиц поиска.           % FIXME: ну написано очень криво. Использование
Использование нескольких потоков позволяет не блокировать работу симуляции на время обработки изображений.  
Далее каждое обновление кадра скрипт считывает изображения с камер и помещает их в память соответствующего 
потока. Поток же работает независимо и обрабатывает каждую следующую пару изображений после готовности 
предыдущей.              % Про виртуальную камеру 
\subsection{Алгоритм устранения искажений}
\label{dewarping}

Распространённые библиотеки машинного зрения предлагают готовые к использованию классы и функции, 
позволяющие устранять искажения fisheye-объективов. Например, в составе MATLAB Computer Vision Toolbox присутствует 
инструмент Camera Calibrator, позволяющий в несколько простых этапов  произвести оценку внутренних и внешних параметров
 камеры. Пользователю надо лишь  загрузить набор изображений, содержащих в себе калибровочный  узор, выбрать модель камеры 
(стандартная или сверхширокоугольная) и запустить автоматический процесс калибровки. По завершению процесса программа 
показывает точность калибровки и может отображать скорректированные изображения. Результаты можно экспортировать в рабочую 
область MATLAB для дальнейшей обработки изображений. Интерфейс программы представлен на рисунке \ref{pic:calibrator}. 

 \addimghere{calibrator}{0.9}{Интерфейс приложения MATLAB Camera Calibrator}{pic:calibrator} 

Однако на практике этот и другие подобные методы весьма ограничены. И MATLAB, и OpenCV устраняют искажения  строго 
в центральной части изображения, не позволяя выбирать  желаемое направление обзора.  Это ограничивает применимость 
существующих инструментов только для копланарного расположения  камер. Поэтому для реализации предлагаемой системы 
стереозрения  был разработан собственный алгоритм устранения искажений. Схема геометрического принципа, лежащего в
 основе этого алгоритма, представлена на рисунке \ref{pic:sweeping}.

\addimghere{projection_sweeping}{0.7}{Схема принципа устранения искажений}{pic:sweeping} 

Цель алгоритма - найти, куда на выходном изображении  с  перспективной проекцией проектируются все пиксели из 
выбранного участка входного сверхширокоугольного изображения. Сделать это можно, выполнив обратное преобразование 
(\ref{eqn:scaramuzza}) для каждого пикселя fisheye-снимка и затем прямое преобразование модели камеры-обскуры 
для получения результирующей проекции. Однако такой подход приведёт к возникновению дефектов из-за несовпадения % TODO:  скорее всего наврал. Надо проконсультироваться 
частот дискретизации двух изображений.  Поэтому вместо этого алгоритм сначала выполняет обратное преобразование  для 
каждого пикселя итогового изображения $\nu$, находя таким образом соответствующую ему точку в системе координат 
камеры $({X_c, Y_c, Z_c})$ 
\begin{equation}
    \label{eq:uv_to_xyz}
    \left[\begin{matrix}x_c\\y_c\\z_c\\\end{matrix}\right] = \left[\begin{matrix} {u*z_c}/f \\  {v*z_c}/f \\ z_c \\\end{matrix}\right],
\end{equation} 
где  $f$ - фокусное расстояние. 

Набор таких точек формирует прямоугольную область $\nu_p$ с центром в точке $O_p$ и является плоскостью изображения
 виртуальной камеры-обскуры  с оптической осью $Z_p$. Поворот точек, входящих в $\nu_p$, с помощью матрицы вращения 
 $\bm{R}$ образует $\nu'_p$ и позволяет таким образом задать направление обзора и ориентацию виртуальной камеры. 
\begin{equation}
    \label{eq:sweeped}
    \left[\begin{matrix}x'_p\\y'_p\\z'_p\\\end{matrix}\right] = \left[\begin{matrix}x_p\\y_p\\z_p\\\end{matrix}\right] \bm{R}.
\end{equation}  
\begin{equation}
    \label{eq:R}
    \bm{R} = \left[\begin{matrix}\cos{\alpha}&-\sin{\alpha}&0\\\sin{\alpha}&\cos{\alpha}&0\\0&0&1\\\end{matrix}\right]\left[\begin{matrix}1&0&0\\0&\cos{\beta}&\sin{\beta}\\0&-\sin{\beta}&\cos{\beta}\\\end{matrix}\right]\left[\begin{matrix}\cos{\gamma}&0&-\sin{\gamma}\\0&1&0\\\sin{\gamma}&0&\cos{\gamma}\\\end{matrix}\right],
\end{equation} 
где координаты с индексом $p$ -  координаты точек соответствующей плоскости $\nu$ в системе камеры, $\alpha, \beta, \gamma$ - углы Эйлера, управляющие  ориентацией виртуальной камеры-обскуры. % TODO: точно ли Эйлера? 

Тогда обратная fisheye-проекция точек из $\nu'_p$ позволяют получить область $\nu_i$ исходного изображения с искомыми пикселями. 
Таким образом, зная как  проектируется каждая точка из $\nu$ в $\nu_i$, возможно  обратно перенести информацию о цвете на итоговое изображение.  

Так как в используемой модели искажения хода луча считаются центрально симметричными и зависят только от его удаления от центра изображения, 
рассмотрим ход падающего луча в координатах $(Z_c, \rho)$, изображённый на рисунке \ref{pic:scara_graph}.

\addimghere{scara_graph}{0.5}{Нахождение обратной проекции для используемой модели}{pic:scara_graph} 

Для модели (\ref{eqn:scaramuzza}) обратная проекция записывается как 
\begin{equation}
    \label{eq:back_scara}
    \left[\begin{matrix}u_i\\v_i\\\end{matrix}\right] = \left[\begin{matrix} \frac{x_c}{\lambda}  \\  \frac{y_c}{\lambda} \\\end{matrix}\right],
\end{equation}  
где $\lambda = \rho_c / \rho_i$ - масштабный коэффициент. 

Для нахождения $\rho_i$ был применён метод последовательных приближений. Блок-схема алгоритма, реализующего обратную проекцию, изображена  
на рисунке  \ref{pic:newton_scheme}.  Сравнение изображений инструмента  калибровки и описанного алгоритма для центральной области изображения
представлено на рисунке \ref{pic:central_pics}. 

\addimghere{remapped_images}{0.8}{Изображения, скорректированные алгоритмом (слева) и MATLAB (справо)}{pic:central_pics} 

Очевидно, весь процесс преобразования $\nu \rightarrow \nu_i$ требует выполнения существенного количества математических операций, что 
негативно сказывается на скорости, с которой алгоритм может обрабатывать изображения в реальном времени. Однако при неизменных параметрах 
модели алгоритм достаточно выполнить лишь один раз, записав результат в таблицу поиска - структуру данных, которая позволяет дальнейшие 
преобразования проводить по уже известным соотношениям между пикселями. Это позволяет применять алгоритм для устранения 
искажений в реальном времени.


\addimghere{flowchart}{0.5}{Блок-схема алгоритма обратной проекции}{pic:newton_scheme}         % Как происходит выпрямление
\subsection{Выводы по второму разделу}

Описан принцип устранения искажений сверхширокоугольных линз с выбором области интереса. Разработан алгоритм нахождения обратной 
проекции для fisheye-изображения. Описано устройство системы стереозрения. Разработана её виртуальная модель в среде Unity вместе с 
алгоритмом передачи изображений с виртуальной камеры в программу обработки изображений.  
Виртуальная модель системы позволяет перейти к её испытаниям.       % Выводы по разделу

\newpage
\section{Экспериментальное исследование системы стереозрения}
Описанная в предыдущей главе виртуальная модель системы стереозрения позволяет проводить 
с ней испытания для оценки работоспособности и сравнения с аналогами в контролируемой среде.            % FIXME: как же плохо       % План экспериментов

\subsection{Оценка качества устранения искажений}

Добавление в виртуальную модель эталонной камеры позволяет оценить качество устранения искажений путём сравнения
её снимков со снимками виртуальной камеры-обскуры модуля устранения искажений. % FIXME: никакой "модуль..." ранее не вводился, нужно переписать

Параметры полинома (\ref{eqn:scaramuzza}) для устранения искажений получены с помощью MATLAB Camera Calibrator и занесены 
в код библиотеки обработки изображений. % FIXME: опять-таки будто про OpenCV речь, надо как-то ссылаться на свой алгоритм\модуль
Снимки после устранения искажений и с эталонной камеры показаны на рисунке \ref{pic:2images_compar}.

\addtwoimghere{fy_house}{0.4}{reg_house}{0.4}{Слева - снимок после устранения искажений; справа - эталонный снимок.}{pic:2images_compar}

На левом изображении заметна большая резкость, вызванная, вероятно, округлениями чисел в процессе проекции. Также присутствует 
затенение по левому краю - следствие аберрации на исходном снимке. В той же области есть малозаметные искажения геометрии. Более 
явно эти дефекты можно увидеть на разностном изображении, представленном на рисунке \ref{pic:difference}. 

\addimghere{difference}{0.5}{Разностное изображение}{pic:difference}

Анализ этого изображения подтверждает различия в резкости и наличие искажений в левой части снимка. Тем не менее, искажения
достаточно несущественны и проявляются лишь близь краёв исходного изображения, что позволяет считать подобные снимки пригодными 
к применению в системе стереозрения.  Кроме того, некоторые эффекты возможно устранить или  сильно ослабить более качественной 
калибровкой камер... % TODO: и...  использованием более качественной оптики? применением системы не так близко к краям? 
         % Сравнение картинок

\subsection{Исследование точности оценки глубины}  % понятие облака точек не введено, поверхность какая? 

В качестве целевой поверхности выбрана виртуальная плоскость с нанесённой на неё нерегулярной текстурой высоко разрешения. Размеры и
положение плоскости относительно камер известно с высокой точностью, что позволяет сравнить результаты стереореконструкции
с реальным положением целевого объекта и оценить ошибку. В качестве метрики оценки выбрано среднее квадратичное отклонение
положения точек от модели плоскости. 

Исследования проведены в MATLAB, снимки получены с помощью виртуальной модели при  одинаковых условиях освещённости и 
постоянных  настройках всех  компонентов системы.  Ширина базы $B$ выбрана равной 1  метру.  Сначала для обеих стереопар 
проведена калибровка по снимкам узора шахматной доски с помощью методики \cite{stereo_calib}, что позволило получить 
внутренние и внешние параметры камер стереопары. % FIXME: ну вообще они у нас как бы есть. Мб сравнить ещё оценки с реальными 
Далее происходит построение карты расхождений методом полу-глобального  сопоставления \cite{SGBM} и 3D-реконструкция сцены. 
Результата реконструкции представлен  в виде облака точек,  изображённого на рисунке \ref{pic:raw_pointcloud}.

\addimghere{pointcloud}{0.7}{Неочищенное облако точек}{pic:raw_pointcloud}
\pdfmargincomment{Три вида с одного раркурса}
% TODO: сюда можно и картинку после обработки поинтклауда \addtwoimghere{}{}{}{}{}
По заданным параметрам строится модель целевой плоскости, а точки за пределами её окрестности отбрасываются. % FIXME: размеры окрестности
Затем скрипт перебирает все оставшиеся точки и рассчитывает среднее квадратичное отклонение по длине нормали от 
точки к плоскости. 
Описанный алгоритм применён к снимкам целевой плоскости на разных расстояниях с эталонной и исследуемой стереопар.
Результаты приведены на графике \ref{plot:MSE_compar}. На графике приведены значения среднеквадратичной ошибки и
дисперсия для исследуемой и эталонной стереопар при разном удалении целевого объекта.

\addimghere{alt1mErrorGraph}{1.0}{График среднеквадратичной ошибки и  дисперсии в зависимости от расстояния для эталонной и исследуемой стереопар}{plot:MSE_compar}

Как видно из аппроксимирующих прямых на графике, среднеквадратичная ошибка оценки поверхности в исследуемой системе
 в 3 раза больше этого показателя для традиционной стереопары. Меньшая точность исследуемой системы связана с влиянием 
на алгоритм стереосопоставления дефектов изображения, описанных в предыдущей секции. Кроме того, с увеличением расстояния
уменьшается разрешение, приходящееся на поверхность, и, соответственно, количество точек с подсчитанной глубиной. 
Тем не менее она остаётся в пределах $3\%$ от реального расстояния.  % FIXME: хммммм, как-то уж больно хорошо. Или  плохо... 10 см на 5 метрах


%         % Оценка равномерности поинтклауда стены
\subsection{Выводы по разделу}

Исследованы снимки, полученные с помощью виртуальных камер. Выполнено их сравнение с эталонными изображениями,  
которое продемонстрировало пригодность снимков для использования в системах стереозрения.

В виртуальную среду добавлена традиционная стереопара для сравнения с разработанной системой. Проведён эксперимент
по оценке качества облака точек, полученного с помощью предлагаемого решения. Результаты можно считать удовлетворительными 
для систем подобного класса. % TODO: \: meh  какого класса? Почему можно считать? 


  % Выводы по разделу

\newpage
\anonsection{Заключение}

В ходе работы рассмотрены существующие системы стереозрения, в особенности использующие информацию
со сверхширокоугольных камер. Изучены модели, описывающие подобные камеры. Рассмотрено ПО для виртуального
 моделирования робототехнических комплексов и систем технического зрения. Выбрано ПО для моделировани и 
разработки системы стереозрения, использующей объективы "рыбий глаз". 

Разработан и описан алгоритм устранения искажений в области интереса и его математическая модель.   % пафосно, конечно.  Так ли хорошо она описана?
Описана возможная конфигурация системы стереозрения, рассмотрен принцип работы её фрагмента. Этот  фрагмент считать
был смоделирован в среде Unity. В процессе размещены и настроены камеры, подготовлены объекты для калибровки и 
дальнейших испытаний системы. Реализована передача изображений с виртуальных камер для обработки алгоритмами
компьютерного зрения. С помощью снимков из виртуальной модели исследовано качество устранения искажений и выполнено
 сравнение точности оценки глубина предлагаемой системы и варианта с обычными камерами. Результаты позволяют рассмотреть
применение системы с реальными камерами и указывают на возможные доработки.  % FIXME: "указывают на доработки" такое себе 

Дальнейшая работа будет сконцентрирована на оптимизации и повышении точности алгоритма устранения искажений,
разработке способа автоматической установки параметров системы при разных конфигурациях камер. После этого можно
будет приступить к испытаниям качества построения карты и локализации с применением описанной системы сначала
с виртуальным, а затем и с реальном роботом. %Заключение-заключение

\newpage
\anonsection{СПИСОК ИСПОЛЬЗОВАННЫХ ИСТОЧНИКОВ}
\printbibliography[heading=none] % печать библиографии , env=thebibliography  title={СПИСОК ИСПОЛЬЗОВАННЫХ ИСТОЧНИКОВ}, heading=bibintoc


\newpage
% Приложения
\appsection{Приложение А} \hypertarget{app-a}{\label{app-a}}

\centering{Сравнение ПО для симуляции}


    \begin{table}[h!]             % TODO: Сыровато. Надо дополнить. Проверить ГОСТовость подписи  FIXME: Таблица не умещается, но вращать не хочется. 
        
        %\caption{Сравнение ПО для симуляции }
        \label{tab:sims}
        \rotatebox{90} {\begin{tabular}{|l|l|l|l|l|}
        \hline
        \textbf{Название} & \textbf{\begin{tabular}[c]{@{}l@{}}Симуляция \\ fisheye-камер\end{tabular}} & \textbf{\begin{tabular}[c]{@{}l@{}}Реалистичное \\ моделирование\end{tabular}} & \textbf{Интеграция кода}                                            & \textbf{Доступность}  \\ \hline
        Gazebo            & Возможна                                                                    & Возможно                                                                     & \begin{tabular}[c]{@{}l@{}}Возможна \\ посредством ROS\end{tabular} & Бесплатно               \\ \hline
        RoboDK            & Нет                                                                         & Затруднено                                                                     & Нет                                                                 & От 145€                \\ \hline
        Webots            & Затруднена                                                                  & Возможно                                                                       & Возможна                                                            & Бесплатно              \\ \hline
        CoppeliaSim       & Затруднена                                                                  & Затруднено                                                                     & Возможна                                                            & Бесплатно              \\ \hline
        NVIDIA Isaac Sim  & Возможна                                                                    & Возможно                                                                       & Возможна                                                            & Бесплатно              \\ \hline
        CARLA             & Затруднена                                                                  & Возможно                                                                       & Возможна                                                            & Бесплатно              \\ \hline
        Unity             & Возможна                                                                    & Возможно                                                                       & Возможна                                                            & Бесплатно              \\ \hline
        \end{tabular}}

    \end{table}
\clearpage


% \appsection{Приложение Б} \hypertarget{app-b}{\label{app-b}}

% \centering{Программный код файла FisheyeDewarper.cpp}
% \begin{english}
% \lstinputlisting[language=C++, numbers=left]{code/FisheyeDewarper.cpp}
% \end{english}

% \clearpage

% \appsection{Приложение В} \hypertarget{app-c}{\label{app-c}}

% \centering{Программный код файла unity\_plugin.cpp}
% \begin{english}
% \lstinputlisting[language=C++, numbers=left]{code/unity_plugin.cpp}
% \end{english}

% \clearpage

% \appsection{Приложение Г} \hypertarget{app-d}{\label{app-d}}

% \centering{Программный код файла Connector.cs}
% \begin{english}
% \lstinputlisting[language=C, numbers=left]{code/Connector.cs}
% \end{english}

% \clearpage

% \appsection{Приложение Д} \hypertarget{app:matlab}{\label{app:matlab}}

% \centering{Программный код файла DisparityPlayground.m}
% \begin{english}
% \lstinputlisting[language=Matlab, numbers=left]{code/DisparityPlayground.m}
% \end{english}

% \clearpage
 % Код скрипта

\end{document} 
%%% Конец документа

\documentclass[a4paper,14pt]{extarticle} % 14й шрифт
%%% Преамбула %%%

\usepackage{fontspec} % XeTeX
\usepackage{xunicode} % Unicode для XeTeX
\usepackage{xltxtra}  % Верхние и нижние индексы
\usepackage{pdfpages} % Вставка PDF
\usepackage{bm}

\usepackage{listings} % Оформление исходного кода
\lstset{
    basicstyle=\small\ttfamily, % Размер и тип шрифта
    breaklines=true, % Перенос строк
    tabsize=2, % Размер табуляции
    literate={--}{{-{}-}}2 % Корректно отображать двойной дефис
}
% Русский язык
\usepackage{polyglossia}
\setdefaultlanguage{russian}
\setotherlanguage{english}

% Шрифты, xelatex
\defaultfontfeatures{Ligatures=TeX}
\setmainfont[Ligatures=TeX]{Times New Roman}
\newfontfamily\cyrillicfont{Times New Roman}[Script=Cyrillic]
%\setsansfont{Liberation Sans} % Альтернатива нью роману, если его нет
\setmonofont{FreeMono} % Моноширинный шрифт для оформления кода
\newfontfamily{\cyrillicfonttt}{FreeMono}


\usepackage{amssymb,amsfonts,amsmath} % Математика
\numberwithin{equation}{section} % Формула вида секция.номер

\usepackage{enumerate} % Тонкая настройка списков
\usepackage{indentfirst} % Красная строка после заголовка
\usepackage{float} % Расширенное управление плавающими объектами
\usepackage{multirow} % Сложные таблицы
\usepackage{adjustbox}
\usepackage{lscape}
\usepackage{rotating}

\usepackage{pdfcomment}     % тудущки и комменты
\pdfcommentsetup{final} % set to 'final' to prevent all comments from showing up   draft
\pdfcommentsetup{color={1.0 1.0 0.0}, open=true}

% Пути к каталогам с изображениями
\usepackage{graphicx} % Вставка картинок и дополнений
\graphicspath{{pics/}{images/userguide/}{images/testing/}{images/infrastructure/}{extra/}{extra/drafts/}}

% Формат подрисуночных записей
\usepackage{chngcntr} 
\counterwithin{figure}{section}

% Гиперссылки
\usepackage{hyperref}
\hypersetup{
    colorlinks, urlcolor={black}, % Все ссылки черного цвета, кликабельные
    linkcolor={black}, citecolor={black}, filecolor={black},
    pdfauthor={Author},
    pdftitle={Title}
}
% Оформление библиографии и подрисуночных записей через точку
\makeatletter
\renewcommand*{\@biblabel}[1]{\hfill#1.}
\renewcommand*\l@section{\@dottedtocline{1}{1em}{1em}}
\renewcommand{\thefigure}{\thesection.\arabic{figure}} % Формат рисунка секция.номер
\renewcommand{\thetable}{\thesection.\arabic{table}} % Формат таблицы секция.номер
\def\redeflsection{\def\l@section{\@dottedtocline{1}{0em}{10em}}}
\makeatother

\renewcommand{\baselinestretch}{1.4} % Полуторный межстрочный интервал
\parindent 1.27cm % Абзацный отступ

\sloppy             % Избавляемся от переполнений
\hyphenpenalty=1000 % Частота переносов
\clubpenalty=10000  % Запрещаем разрыв страницы после первой строки абзаца
\widowpenalty=10000 % Запрещаем разрыв страницы после последней строки абзаца

% Отступы у страниц
\usepackage{geometry}
\geometry{left=3cm}
\geometry{right=1.5cm}
\geometry{top=2cm}
\geometry{bottom=2cm}

% Списки
\usepackage{enumitem}
\setlist[enumerate,itemize]{leftmargin=0mm,itemindent = 20mm} % Отступы в списках

\makeatletter
    \AddEnumerateCounter{\asbuk}{\@asbuk}{м)}
\makeatother
\setlist{nolistsep} % Нет отступов между пунктами списка
\renewcommand{\labelitemi}{--} % Маркет списка --
\renewcommand{\labelenumi}{\asbuk{enumi})} % Список второго уровня
\renewcommand{\labelenumii}{\arabic{enumii})} % Список третьего уровня
                    % Глубина оглавления, до subsubsection
\usepackage{scrextend}


% Нумерация страниц по центру снизу
\usepackage{fancyhdr}
\pagestyle{plain}
\fancyhf{}
\fancyhead[R]{\textrm{\thepage}}
\fancyheadoffset{0mm}
\fancyfootoffset{0mm}
\setlength{\headheight}{17pt}
\renewcommand{\headrulewidth}{0pt}
\renewcommand{\footrulewidth}{0pt}
\fancypagestyle{plain}{ 
    \fancyhf{}
    \rhead{\thepage}
}
% Содержание
\usepackage{tocloft}
\renewcommand{\cftaftertoctitle}{
    \thispagestyle{plain}}     % Убирает номер страницы
\renewcommand{\cfttoctitlefont}{\hspace{0.38\textwidth}\MakeTextUppercase} % СОДЕРЖАНИЕ
\renewcommand{\cftsecfont}{\hspace{0pt}}            % Имена секций в содержании не жирным шрифтом
\renewcommand\cftsecleader{\cftdotfill{\cftdotsep}} % Точки для секций в содержании
\renewcommand\cftsecpagefont{\mdseries}             % Номера страниц не жирные
\setcounter{tocdepth}{3}        

% GOST - biblatex
\usepackage[% 
bibstyle=gost-numeric, %подключение одного из четырех главных стилей biblatex-gost 
sorting=none, %тип сортировки в библиографии
doi=false,  
]{biblatex}
\addbibresource{refs.bib}

% Формат подрисуночных надписей
\RequirePackage{caption}
\DeclareCaptionLabelSeparator{defffis}{ -- } % Разделитель
\captionsetup[figure]{justification=centering, labelsep=defffis, format=plain, belowskip=-15pt} % Подпись рисунка по центру
\captionsetup[table]{justification=raggedright, labelsep=defffis, format=plain, singlelinecheck=false} % Подпись таблицы слева
\addto\captionsrussian{\renewcommand{\figurename}{Рисунок}} % Имя фигуры

% Пользовательские функции
\newcommand{\addimg}[4]{ % Добавление одного рисунка
    \begin{figure}
        \centering
        \includegraphics[width=#2\linewidth]{#1}
        \caption{#3} \label{#4}
    \end{figure}
}
\newcommand{\addimghere}[4]{ % Добавить рисунок непосредственно в это место
    \begin{figure}[H]
        \centering
        \includegraphics[width=#2\linewidth]{#1}
        \caption{#3} \label{#4}
    \end{figure}
}
\newcommand{\addtwoimghere}[6]{ % Вставка двух рисунков
    \begin{figure}[H]
        \centering
        \includegraphics[width=#2\linewidth]{#1}
        \hspace{1cm}
        \includegraphics[width=#4\linewidth]{#3}
        \caption{#5} \label{#6}
    \end{figure}
}
\newcommand{\addimgapp}[2]{ % Это костыль для приложения Б
    \begin{figure}[H]
        \centering
        \includegraphics[width=1\linewidth]{#1}
        \caption*{#2}
    \end{figure}
}

% Заголовки секций в оглавлении в верхнем регистре
\usepackage{textcase}
\makeatletter
\let\oldcontentsline\contentsline
\def\contentsline#1#2{
    \expandafter\ifx\csname l@#1\endcsname\l@section
        \expandafter\@firstoftwo
    \else
        \expandafter\@secondoftwo
    \fi
    {\oldcontentsline{#1}{#2}}          % {\oldcontentsline{#1}{\MakeTextUppercase{#2}}} 
    {\oldcontentsline{#1}{#2}}
}
\makeatother

% Оформление заголовков
\usepackage[compact,explicit]{titlesec}
\titleformat{\section}{\large\bfseries}{}{0mm}{\centering{\thesection\quad\textbf{#1}} }  % \titleformat{\section}{}{}{12.5mm}{\centering{\thesection\quad\MakeTextUppercase{#1}}\vspace{1.5em}}
\titleformat{\subsection}[block]{\bfseries}{}{12.5mm}{\thesubsection\quad\textbf{#1}}          %
\titleformat{\subsubsection}[block]{\vspace{1em}\normalsize}{}{12.5mm}{\thesubsubsection\quad#1\vspace{1em}}
\titleformat{\paragraph}[block]{\normalsize}{}{12.5mm}{\MakeTextUppercase{#1}}


% Секции без номеров (введение, заключение...), вместо section*{}
\newcommand{\anonsection}[1]{
    \phantomsection % Корректный переход по ссылкам в содержании
    \paragraph{\centerline{{#1}}\vspace{0em}}
    \addcontentsline{toc}{section}{\uppercase{#1}}
}

% Секции для приложений
\newcommand{\appsection}[1]{
    \phantomsection
    \paragraph{\centerline{{#1}}}
    \addcontentsline{toc}{section}{\uppercase{#1}}
}

% Библиография: отступы и межстрочный интервал
% \makeatletter
% \defbibenvironment{thebibliography}
% {
%     \list{\@biblabel{\@arabic\c@enumiv}}
%         {\settowidth\labelwidth{\@biblabel{#1}}
%         \leftmargin\labelsep
%         \itemindent 16.7mm
%         \@openbib@code
%         \usecounter{enumiv}
%         \let\p@enumiv\@empty
%         \renewcommand\theenumiv{\@arabic\c@enumiv}}
%         }
%     \setlength{\itemsep}{0pt}
%     }
% \makeatother

\setcounter{page}{2} % Начало нумерации страниц % Подключаем преамбулу

%%% Начало документа
\begin{document}

\include{include/titlepage.tex}
%\includepdf{pz} % Пояснительная записка
%\includepdf[pages={1,2}]{task} % Задание на диплом 
\include{include/referat.tex}

\tableofcontents % Содержание 
\clearpage

\anonsection{введение}

Получение трёхмерной структуры пространства по стереоснимкам - это задача, первые решения которой
были получены десятилетия назад. Ранние работы фокусировались в основном на способах поиска соответствий
и геометрических основах, лежащих в основе процесса. Существенный объём научной работы продолжает
 производиться в области стереозрения и по сей день. Был достигнут заметный прогресс в повышении точности 
 результатов и понижении вычислительных мощностей, требуемых для их достижения, однако эти области остаются 
 фокусом исследований. 

Улучшение точности и производительности алгоритмов является нетривиальной задачей. На точность 
полученных результатов оказывает влияние нехватка информации, вызванная заслонением объектов, наличием наклонных
плоскостей и другими факторами, влияющими на сложность восстановления трёхмерных объектов. Разрешение
сенсоров также растёт с каждым годом, увеличивая вычислительную сложность поиска соответствий на кадрах с 
каждой камеры. Таким образом, исследователи в области стереозрения пытаются найти компромисс между этими
 двумя характеристиками. Однако для каждого конкретного алгоритма этот компромисс может быть смещён в 
 ту или иную сторону. \cite{Hartley2004}
\section{Обзор опыта построения систем стереозрения с использованием сверхширокоугольных камер}
\section{Обзор алгоритмов стереозрения / моделей fisheye камер / оснащённости роботов соответствующими камерами}

TODO: в этой секции стоит провести обзор. Пока не очень понятно, чего конкретно. Можно наверное и всего понемногу
\subsection{Обоснование выбора ПО}

Разработку и первоначальные испытания алгоритма стереозрения целесообразно проводить в виртуальной среде. Это позволяет значительно упростить разработку, 
так как уменьшает время на проверку гипотез и расходы на реальное оборудование, особенно в случае неудачных испытаний. Из-за этих факторов виртуальное моделирование 
в робототехнике приобрело широкое распространение и активно применяется, например, для разработки систем локализации и навигации беспилотного транспорта \cite{}. 
Возросшее качество компьютерной графики к тому же позволило моделировать реалистичное окружение, что особенно важно при работе с системами технического зрения. 

Для разработки алгоритма, описанного в этой работе, нужна виртуальная среда, в которой можно симулировать несколько широкоугольных камер и настраивать их параметры, 
легко интегрировать алгоритмы технического зрения и создать окружение, приближенное к тому, в котором будет работать алгоритм. На данный момент исследователю доступен 
широкий выбор программного обеспечения, подходящего для этой задачи. В таблице \ref{tab:sims} представлено сравнение имеющихся предложений по основным изложенным выше 
требованиям.

\begin{table}[]             % TODO: Сыровато. Надо дополнить. Проверить ГОСТовость подписи  FIXME: Таблица не умещается, но вращать не хочется. 
    \caption{Сравнение ПО для симуляции }
    \label{tab:sims}
    \begin{tabular}{|l|l|l|l|l|l|}
    \hline
    \textbf{Название} & \textbf{\begin{tabular}[c]{@{}l@{}}Симуляция \\ fisheye-камер\end{tabular}} & \textbf{\begin{tabular}[c]{@{}l@{}}Реалистичное \\ моделирование\end{tabular}} & \textbf{Интеграция кода}                                            & \textbf{Доступность} & \textbf{Примечания} \\ \hline
    Gazebo            & Возможна                                                                    & Затруднено                                                                     & \begin{tabular}[c]{@{}l@{}}Возможна \\ посредством ROS\end{tabular} & Бесплатно            &                     \\ \hline
    RoboDK            & Нет                                                                         & Затруднено                                                                     & Нет                                                                 & От 145€              &                     \\ \hline
    Webots            & Затруднена                                                                  & Возможно                                                                       & Возможна                                                            & Бесплатно            &                     \\ \hline
    CoppeliaSim       & Затруднена                                                                  & Затруднено                                                                     & Возможна                                                            & Бесплатно            &                     \\ \hline
    NVIDIA Isaac Sim  & Возможна                                                                    & Возможно                                                                       & Возможна                                                            & Бесплатно            &                     \\ \hline
    CARLA             & Затруднена                                                                  & Возможно                                                                       & Возможна                                                            & Бесплатно            &                     \\ \hline
    Unity             & Возможна                                                                    & Возможно                                                                       & Возможна                                                            & Бесплатно            &                     \\ \hline
                      &                                                                             &                                                                                &                                                                     &                      &                     \\ \hline
    \end{tabular}
\end{table}

По результатам оценки собранные сведений было принято решение проводить разработку в симуляторе Unity. Он позволяет подробно настраивать камеру и эмулировать fisheye-объектив, 
строить реалистичные сцены благодаря свободному импорту моделей, а при программировании в симуляторе можно использовать сторонние программы в виде динамически подключаемых библиотек. 
По функционалу так же подходит NVIDIA Isaac Sim, но от него пришлось отказаться из-за высоких системных требований и новизны продукта.     % TODO: звучит тупо

Разрабатываемое решение должно иметь возможность внедрения в ПО робота, % TODO: не совсем ПО всё-таки
поэтому должно реализовываться на одном из популярных и быстродейственных языков программирования. Учитывая необходимость интеграции с Unity и потребность использовать популярные библиотеки
, был выбран язык C++.  Другим важным фактором является основная библиотека обработки изображений. В качестве основы для программной части была выбрана библиотека OpenCV, являющаяся стандартом 
при разработке систем технического зрения. Она доступна к использованию со множеством языков программирования, но наилучшую производительность показывает именно с C++ \cite{}.                              % мб есть сравнение
\subsection{Выводы по главе}

Обзор современных моделей сверхширокоугольных камер позволил выбрать 
наиболее точную и удобную для калибровки. Был осуществлён обзор существующих
систем стереозрения, применяющих камеры типа "рыбий глаз" с целью ознакомления 
с мировым опытом. Было принято решение разрабатывать и тестировать предлагаемую 
систему стереозрения с применением виртуального  моделирования. 

Для моделирования выбрана среда разработки Unity. Для обработки изображений
с камер выбрана библиотека OpenCV для языка программирования C++. 

\newpage
\section{Система стереозрения}
\subsection{Алгоритм устранения искажений}

Как уже упоминалось в секции \ref{camera_model}, существующие способы устранения радиальных искажений не
позволяют работать близко к краям изображений, поэтому для реализации предлагаемой системы стереозрения     % FIXME: "работать"
был разработан алгоритм устранения искажений на основе модели \cite{scaramuzza}. Схема геометрического
принципа, лежащего в основе этого алгоритма, представлена на рисунке \ref{pic:sweeping}.

\addimg{projection_sweeping}{0.7}{Схема принципа устранения искажений}{pic:sweeping} 
Цель алгоритма - найти, куда на выходном изображении проектируются все пиксели из выбранного участка входного 
сверхширокоугольного изображения. Сделать это можно, выполнив обратное преобразование \ref{eqn:scaramuzza} для каждого пикселя 
fisheye-снимка и затем прямое преобразование модели камеры-обскуры для получения результирующей проекции. Однако такой подход 
приведёт к возникновению дефектов из-за несовпадения частот дискретизации двух изображений.                 % TODO:  скорее всего наврал. Надо проконсультироваться 
Поэтому вместо этого алгоритм выполняет обратное преобразование  для каждого пикселя итогового изображения $\nu$, 
находя таким образом соответствующую ему точку в системе координат камеры $({X_c, Y_c, Z_c})$ 
\begin{equation}
    \label{eq:uv_to_xyz}
    \left[\begin{matrix}x_c\\y_c\\z_c\\\end{matrix}\right] = \left[\begin{matrix} {(u+c_x)*z_c}/f \\  {(v+c_y)*z_c}/f \\ z_c \\\end{matrix}\right],
\end{equation} 
где $c_x, c_y$ - координаты центра изображения; $f$ - фокусное расстояние. 

Набор таких точек формирует прямоугольник $\nu_p$ с центром в точке $O_p$ и является виртуальной камерой-обскурой        % FIXME: прямоугольник?
с оптической осью $Z_p$. Поворот точек, входящих в $\nu_p$, с помощью матрицы вращения $\bm{R}$ образует $\nu'_p$ и
 позволяет таким образом задать направление обзора и ориентацию виртуальной камеры. 
\begin{equation}
    \label{eq:R}
    \bm{R} = \left[\begin{matrix}\cos{\alpha}&-\sin{\alpha}&0\\\sin{\alpha}&\cos{\alpha}&0\\0&0&1\\\end{matrix}\right]\left[\begin{matrix}1&0&0\\0&\cos{\beta}&\sin{\beta}\\0&-\sin{\beta}&\cos{\beta}\\\end{matrix}\right]\left[\begin{matrix}\cos{\gamma}&0&-\sin{\gamma}\\0&1&0\\\sin{\gamma}&0&\cos{\gamma}\\\end{matrix}\right],
\end{equation} 
где $\alpha, \beta, \gamma$ - углы Эйлера. % TODO: точно ли Эйлера? 
\begin{equation}
    \label{eq:sweeped}
    \left[\begin{matrix}x'_p\\y'_p\\z'_p\\\end{matrix}\right] = \left[\begin{matrix}x_p\\y_p\\z_p\\\end{matrix}\right] \bm{R}.
\end{equation}  

Тогда обратная fisheye-проекция точек из $\nu'_p$ позволяют получить область $\nu_i$ исходного изображения с искомыми пикселями. 
Таким образом, зная как геометрически проектируется каждая точка из $\nu$ в $\nu_i$, можно перенести информацию о цвете и получить 
изображение с устранёнными радиальными искажениями в любой части поля зрения камеры. 

Однако модель \ref{eqn:scaramuzza} не даёт явно выраженной обратной проекции. % FIXME: "явно выраженной" 
Так как в используемой модели искажения в точке зависят только от её удаления от центра изображения, рассмотрим ход падающего луча 
в координатах $(Z_c, \rho)$

\addimghere{scara_graph}{0.5}{}{pic:scara_graph} 


   % Как происходит выпрямление
\subsection{Описание системы стереозрения}

Предлагаемая система стереозрения   % Описание системы: Какие этапы, элементы и тд
\subsection{Виртуальное моделирование системы}

Описанная система была смоделирована в среде Unity, её внешний вид представлен на рисунке \ref{pic:unity_model}. Для 
моделирования камеры "рыбий глаз" использовался ад-дон Dome Projector из магазина Unity Asset Store.    % TODO: проверить все названия
Он позволяет моделировать сверхширокоугольные объективы с разным углом зрения в эквидистантной проекции \cite{}. % TODO: тут бы сослаться на их работку какую-нибудь
При этом с точки зрения других искажений, не относящихся к моделированию ___, снимки с этой камеры получаются идеальными. % FIXME: моделированию чего? 
Окно настроек камеры представлено на рисунке \ref{pic:camera_settings}.
\begin{figure}[H]
    \begin{center}
        \includegraphics[scale=0.5]{pics/camera_settings.png}                                                                                            %TODO: перерисовать схему?
        \caption{Окно настроек виртуальной камеры в Unity}
        \label{pic:camera_settings}
    \end{center}
\end{figure}
Для виртуальной камеры доступны настройки угла зрения и виньетки по краям изображения,  % FIXME: виньетки -> абберации ??
а также различных параметров рендеринга, влияющих на качество получаемого изображения.   % Моделирование системы 
\subsection{Передача изображений с виртуальных камер для обработки}
Unity в качестве основного языка программирования использует C\# и не позволяет использовать 
выбранную библиотеку технического зрения OpenCV напрямую. Однако как уже упоминалось в секции 
\ref{sec:software}, данная среда позволяет интегрировать плагины, в том числе написанные на 
других языках программирования, в виде динамически подключаемых библиотек (DLL). Именно этот 
подход был использован для передачи изображений - была разработана библиотека с функциями для 
обработки снимков, которая компилировалась отдельно и затем импортировалась в Unity. 

Для реализации описанного в этой работе принципа и упрощения разработки в библиотеке были реализованы 
следующие функции:
\begin{itemize}
    \item initialize - выделяет память под нужные структуры, заполняет таблицу поиска и создаёт окна 
    для отображения будущих изображений.
    \item getImages - передаёт изображения в библиотеку и производит обработку изображений в 
    соответствии с аргументами. 
    \item takeScreenshot - производит ту же обработку входного изображения, что и предыдущая функция,
     но результат сохраняет в файл. 
    \item processImage - передаёт обратно в скрипт уже обработанные изображения для удобного отображения
    в интерфейсе Unity. 
    \item terminate - высвобождает память по завершении работы симуляции.      
\end{itemize}
Обработка, упомянутая в пунктах с функциями getImages и takeScreenshot, заключается в конвертации 
цветового пространства изображений (из RGBA, используемого в Unity, в BGR, используемый в OpenCV), 
горизонтальном зеркальном отображении (из-за разного положения точки отсчёта координат пикселей)
и, наконец, устранение искажений в области интереса. Так как для наилучшей работы стереосопоставления
снимки должны быть синхронизированы (то есть получены камерами в один момент времени), обработка 
производится сразу для двух изображений. Программный код файлов, входящих в динамически 
подключаемую библиотеку, представлен в приложении \ref{app:dewarper}.

Также был написан специальный скрипт, который управляет процессом взаимодействия симуляции и библиотеки, 
его код приведён в приложении \ref{app:script}. Эта программа имеет окно настроек, которое позволяет задать
используемые камеры и элементы управления параметрами, выбрать область интереса и \dots Внешний вид этого 
окна представлен на рисунке \ref{pic:connec_inter}.
\begin{figure}[H]
    \begin{center}
        \includegraphics[scale=0.5]{pics/connector_settings.png}                                                                                            
        \caption{Внешний вид окна настроек скрипта}
        \label{pic:connec_inter}
    \end{center}
\end{figure}
Сразу после запуска симуляции происходит передача настроек скриптом в библиотеку для построения таблиц поиска           % FIXME: ну написано очень криво
и запуск дополнительных вычислительных потоков для обработки кадров с каждой пары камер. Использование
нескольких потоков позволяет не блокировать работу симуляции на время обработки изображений.  Далее каждое обновление
кадра скрипт считывает изображения с камер и помещает их в память соответствующего потока. Поток же работает независимо 
и обрабатывает каждую следующую пару изображений после готовности карты глубины предыдущей.   % Про виртуальную камеру 
\subsection{Выводы по второму разделу}

Описан принцип устранения искажений сверхширокоугольных линз с выбором области интереса. Разработан алгоритм нахождения обратной 
проекции для fisheye-изображения. Описано устройство системы стереозрения. Разработана её виртуальная модель в среде Unity вместе с 
алгоритмом передачи изображений с виртуальной камеры в программу обработки изображений.  
Виртуальная модель системы позволяет перейти к её испытаниям.  % Выводы по главе

\newpage
\section{Экспериментальное исследование системы стереозрения}
Описанная в предыдущей главе виртуальная модель системы стереозрения позволяет проводить 
с ней испытания для оценки работоспособности и сравнения с аналогами в контролируемой среде.            % FIXME: как же плохо  % План экспериментов

\subsection{Исследование точности оценки глубины}  % понятие облака точек не введено, поверхность какая? 

В качестве целевой поверхности выбрана виртуальная плоскость с нанесённой на неё нерегулярной текстурой высоко разрешения. Размеры и
положение плоскости относительно камер известно с высокой точностью, что позволяет сравнить результаты стереореконструкции
с реальным положением целевого объекта и оценить ошибку. В качестве метрики оценки выбрано среднее квадратичное отклонение
положения точек от модели плоскости. 

Исследования проведены в MATLAB, снимки получены с помощью виртуальной модели при  одинаковых условиях освещённости и 
постоянных  настройках всех  компонентов системы.  Ширина базы $B$ выбрана равной 1  метру.  Сначала для обеих стереопар 
проведена калибровка по снимкам узора шахматной доски с помощью методики \cite{stereo_calib}, что позволило получить 
внутренние и внешние параметры камер стереопары. % FIXME: ну вообще они у нас как бы есть. Мб сравнить ещё оценки с реальными 
Далее происходит построение карты расхождений методом полу-глобального  сопоставления \cite{SGBM} и 3D-реконструкция сцены. 
Результата реконструкции представлен  в виде облака точек,  изображённого на рисунке \ref{pic:raw_pointcloud}.

\addimghere{pointcloud}{0.7}{Неочищенное облако точек}{pic:raw_pointcloud}
\pdfmargincomment{Три вида с одного раркурса}
% TODO: сюда можно и картинку после обработки поинтклауда \addtwoimghere{}{}{}{}{}
По заданным параметрам строится модель целевой плоскости, а точки за пределами её окрестности отбрасываются. % FIXME: размеры окрестности
Затем скрипт перебирает все оставшиеся точки и рассчитывает среднее квадратичное отклонение по длине нормали от 
точки к плоскости. 
Описанный алгоритм применён к снимкам целевой плоскости на разных расстояниях с эталонной и исследуемой стереопар.
Результаты приведены на графике \ref{plot:MSE_compar}. На графике приведены значения среднеквадратичной ошибки и
дисперсия для исследуемой и эталонной стереопар при разном удалении целевого объекта.

\addimghere{alt1mErrorGraph}{1.0}{График среднеквадратичной ошибки и  дисперсии в зависимости от расстояния для эталонной и исследуемой стереопар}{plot:MSE_compar}

Как видно из аппроксимирующих прямых на графике, среднеквадратичная ошибка оценки поверхности в исследуемой системе
 в 3 раза больше этого показателя для традиционной стереопары. Меньшая точность исследуемой системы связана с влиянием 
на алгоритм стереосопоставления дефектов изображения, описанных в предыдущей секции. Кроме того, с увеличением расстояния
уменьшается разрешение, приходящееся на поверхность, и, соответственно, количество точек с подсчитанной глубиной. 
Тем не менее она остаётся в пределах $3\%$ от реального расстояния.  % FIXME: хммммм, как-то уж больно хорошо. Или  плохо... 10 см на 5 метрах


%  % Оценка равномерности поинтклауда стены
\subsection{Выводы по разделу}

Исследованы снимки, полученные с помощью виртуальных камер. Выполнено их сравнение с эталонными изображениями,  
которое продемонстрировало пригодность снимков для использования в системах стереозрения.

В виртуальную среду добавлена традиционная стереопара для сравнения с разработанной системой. Проведён эксперимент
по оценке качества облака точек, полученного с помощью предлагаемого решения. Результаты можно считать удовлетворительными 
для систем подобного класса. % TODO: \: meh  какого класса? Почему можно считать? 


  % Выводы по главе

\newpage
\anonsection{Заключение}
%\include{} %Заключение-заключение

\newpage
\anonsection{СПИСОК ИСПОЛЬЗОВАННЫХ ИСТОЧНИКОВ}
\printbibliography[heading=none] % печать библиографии   title={СПИСОК ИСПОЛЬЗОВАННЫХ ИСТОЧНИКОВ}, heading=bibintoc


\newpage
% Приложения
\appsection{Приложение А} \hypertarget{app-a}{\label{app-a}}

\centering{Сравнение ПО для симуляции}


    \begin{table}[h!]             % TODO: Сыровато. Надо дополнить. Проверить ГОСТовость подписи  FIXME: Таблица не умещается, но вращать не хочется. 
        
        %\caption{Сравнение ПО для симуляции }
        \label{tab:sims}
        \rotatebox{90} {\begin{tabular}{|l|l|l|l|l|}
        \hline
        \textbf{Название} & \textbf{\begin{tabular}[c]{@{}l@{}}Симуляция \\ fisheye-камер\end{tabular}} & \textbf{\begin{tabular}[c]{@{}l@{}}Реалистичное \\ моделирование\end{tabular}} & \textbf{Интеграция кода}                                            & \textbf{Доступность}  \\ \hline
        Gazebo            & Возможна                                                                    & Возможно                                                                     & \begin{tabular}[c]{@{}l@{}}Возможна \\ посредством ROS\end{tabular} & Бесплатно               \\ \hline
        RoboDK            & Нет                                                                         & Затруднено                                                                     & Нет                                                                 & От 145€                \\ \hline
        Webots            & Затруднена                                                                  & Возможно                                                                       & Возможна                                                            & Бесплатно              \\ \hline
        CoppeliaSim       & Затруднена                                                                  & Затруднено                                                                     & Возможна                                                            & Бесплатно              \\ \hline
        NVIDIA Isaac Sim  & Возможна                                                                    & Возможно                                                                       & Возможна                                                            & Бесплатно              \\ \hline
        CARLA             & Затруднена                                                                  & Возможно                                                                       & Возможна                                                            & Бесплатно              \\ \hline
        Unity             & Возможна                                                                    & Возможно                                                                       & Возможна                                                            & Бесплатно              \\ \hline
        \end{tabular}}

    \end{table}
\clearpage


% \appsection{Приложение Б} \hypertarget{app-b}{\label{app-b}}

% \centering{Программный код файла FisheyeDewarper.cpp}
% \begin{english}
% \lstinputlisting[language=C++, numbers=left]{code/FisheyeDewarper.cpp}
% \end{english}

% \clearpage

% \appsection{Приложение В} \hypertarget{app-c}{\label{app-c}}

% \centering{Программный код файла unity\_plugin.cpp}
% \begin{english}
% \lstinputlisting[language=C++, numbers=left]{code/unity_plugin.cpp}
% \end{english}

% \clearpage

% \appsection{Приложение Г} \hypertarget{app-d}{\label{app-d}}

% \centering{Программный код файла Connector.cs}
% \begin{english}
% \lstinputlisting[language=C, numbers=left]{code/Connector.cs}
% \end{english}

% \clearpage

% \appsection{Приложение Д} \hypertarget{app:matlab}{\label{app:matlab}}

% \centering{Программный код файла DisparityPlayground.m}
% \begin{english}
% \lstinputlisting[language=Matlab, numbers=left]{code/DisparityPlayground.m}
% \end{english}

% \clearpage
 % Код скрипта

\end{document} 
%%% Конец документа

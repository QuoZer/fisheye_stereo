\include{settings}

\begin{document}	% начало документа

\include{titlepage}


% Содержание
\tableofcontents
\newpage


\section{Введение}

Получение трёхмерной структуры пространства по стереоснимкам - это задача, первые решения которой
были получены десятилетия назад. Ранние работы фокусировались в основном на способах поиска соответствий
и геометрических основах, лежащих в основе процесса. Существенный объём научной работы продолжает
 производиться в области стереозрения и по сей день. Был достигнут заметный прогресс в повышении точности результатов и понижении вычислительных мощностей, требуемых для 
их достижения. 

Улучшение точности и производительности алгоритмов является нетривиальной задачей. На точность 
полученных результатов оказывает влияние нехватка информации, вызванная заслонением объектов, наличием наклонных
плоскостей и другими факторами, влияющими на сложность восстановления трёхмерных объектов. Разрешение
сенсоров также растёт с каждым годом, увеличивая вычислительную сложность поиска соответствий на кадрах с 
каждой камеры. Таким образом, современные алгоритмы стереозрения пытаются определить компромисс между этими
 двумя характеристиками. Однако для каждого конкретного алгоритма этот компромисс может быть смещён в 
 ту или иную сторону. 



Первые работы (1970-1980гг) выполнялись преимущественно в ...


Остальная часть статьи организована следующим образом:
...

\section{Принцип стереозрения}

Несмотря на существование разных алгоритмов стереозрения, все они реализуют общий принцип. Задача стереозрения - 
получение состоит в использовании двух или нескольких камер для получения данных о дальности до объекта. [elib] 
Как правило, система стереозрения состоит из двух камер, наблюдающих сцену с разных точек. Фундаментальная основа принципа
заключается в предположении, что каждой точке в пространстве соответствует уникальная пара пикселей на снимках с двух камер.  

При этом к камерам предъявляются некоторые требования.
\begin{itemize}
	\item Камеры откалиброваны. Это значит, что известны внутренние (оптические) и внешние (расположение камер в пространстве) параметры камер. 
	\item Ректификация. Подразумевает выравнивание изображения с обеих камер по строкам.  % Мб подробнее расписать  
\end{itemize}

\begin{figure}[H]
	\begin{center}
		\includegraphics[scale=0.7]{pics/epipolar geometry.png}
		\caption{Эпиполярная геометрия} 
		\label{pic:epipol} % название для ссылок внутри кода
	\end{center}
\end{figure}

Таким образом, соблюдение указанных выше требований позволяет использовать следующий геометрический принцип. Пусть имеются две камеры, как изображено 
на \cite{pic:epipol}. C — центр первой камеры, C' — центр второй камеры. Точка пространства X 
проецируется в x на плоскость изображения левой камеры и в x' на плоскость изображения правой камеры. Прообразом точки x на изображении левой 
камеры является луч xX. Этот луч проецируется на плоскость второй камеры в прямую l', называемую эпиполярной линией. Образ точки X на плоскости 
изображения второй камеры обязательно лежит на эпиполярной линии l'.

Таким образом, каждой точке x на изображении левой камеры соответствует эпиполярная линия l' на изображении правой камеры. При этом пара для x на 
изображении правой камеры может лежать только на соответствующей эпиполярной линии. Аналогично, каждой точке x' на правом изображении соответствует 
эпиполярная линия l на левом.

Далее с помощью точек x и x' возможно посчитать смещения каждого пикселя одного изображения относительно другого, что даёт карту смещений (disparity map). 
Очевидно, что смещения будут подсчитаны только для точек, видимых обеими камерами. 

На практике 




@Book{Hartley2004,
    author = "Hartley, R.~I. and Zisserman, A.",
    title = "Multiple View Geometry in Computer Vision",
    edition = "Second",
    year = "2004",
    publisher = "Cambridge University Press, ISBN: 0521540518"
}

\subsection{Список}

\begin{itemize}
\item первый элемент списка
\item второй элемент списка
\end{itemize}


\subsection{Картинка}

\begin{figure}[H]
	\begin{center}
		\includegraphics[scale=0.7]{pics/sample}
		\caption{название картинки} 
		\label{pic:pic_name} % название для ссылок внутри кода
	\end{center}
\end{figure}


Новый параграф

\noindent Новый параграф с принудительно выключенным отступом


\subsection{Таблица}

\begin{table}[H]
	\begin{center}
		\begin{tabular}{|l|l|}
			\hline
			top left & top right\\ \hline
			bot left & bot right\\ \hline
		\end{tabular}
		\caption{ Название таблицы}
		\label{tabular:tab_examp}
	\end{center}
\end{table}

\section{Выводы}
\LaTeX\ удобен для создания отчётов, так как сам следит за нумерацией таблиц, рисунков, листингов и отсылок к ним (так, например, здесь всегда будет указан номер рисунка "sample" не зависимо от того, какой он (1,2 или другой) - это рисунок \ref{pic:pic_name}). Не менее важно что весь документ оформлен в едином стиле, а исходные материалы подключаются к отчёту, а не хранятся в нём. Всё это позволяет легко получить качественный отчёт без дополнительных трат на его офрмление.

Исключения, пожалуй, составляют таблицы, так как их значительно сложнее создавать кодом, нежели в графическом редакторе. Но здесь никто не запрещает использовать визуальные средства создания таблиц для \LaTeX\ .
\end{document}
